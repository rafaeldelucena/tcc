\textoResumo {Redes de sensores são utilizadas para a captação, processamento de informação e atuação sobre um ambiente, tornando-as importantes para controle, telemetria e rastreamento de sistemas. Os nós das redes geralmente são computadores e rádios simplificados, que possuem restrições de memória, processamento, energia e comunicação, mas um custo relativamente baixo de equipamentos, tornando interessante a implantação destes sistemas. O protocolo HTTP foi desenvolvido pensado em computadores de propósito geral, onde essas restrições não existem. Um protocolo leve como CoAP pode tornar viável a criação de aplicações web em redes de sensores sem fio por um baixo custo.
É proposto uma infraestrutura de comunicação entre redes de sensores sem fio e a Internet, utilizando protocolos leves entre os nós sensores e um gateway que utiliza GPRS para áreas sem acesso à WIFI, aproveitando a vasta abrangência da tecnologia de telefonia. Com a Utilização do CoAP é esperado uma redução de consumo de energia e memória, em relação a outros protocolos de aplicação existentes.}
