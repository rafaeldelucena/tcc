Este trabalho prop\~oes a implementa\c{c}\~ao de uma biblioteca que utiliza a camada UDP do EPOS para dar suporte ao protocolo CoAP.

O trabalho tamb\'em consiste na implementa\c{c}\~ao de uma aplica\c{c}\~ao para gateway GPRS/Zigbee utilizando o EPOS e um componente de hardware que ser\'a acoplado ao EposMoteII. Este componente esta sendo desenvolvido em paralelo por um colega de laborat\'orio.

O desenvolvimento da aplica\c{c}\~ao no EPOS que ser\'a respons\'avel pelo roteamento de mensagens para Internet utiliza a tecnologia GPRS, provida por um m\'odulo GSM/GPRS da Quectel o M95.

As principais fun\c{c}\~oes deste gateway s\~ao receber os dados da rede de sensores e encaminh\'a-las para um servidor remoto que armazenar\'a essas informa\c{c}\~oes e exibir\'a de forma conveniente para o usu\'ario final.

As fun\c{c}\~oes a serem desenvolvidas na aplica\c{c}\~ao do gateway s\~ao:

\begin{itemize}[noitemsep,topsep=0pt,parsep=0pt,partopsep=0pt]
    \item Configura\c{c}\~ao SMS
    \item Envia mensagem SMS
    \item Recebe mensagem SMS
    \item Configura\c{c}\~ao contexto PDP
    \item Configura\c{c}\~ao GPRS
    \item Configura\c{c}\~ao TCP/IP
    \item Manter uma conex\~ao TCP/IP 
\end{itemize}

\section{Metas}
Entregar um m\'odulo simplificado do procotolo CoAP no primeiro semestre e testar as funcionalidades no modem GSM/GPRS.
Desenvolver a aplica\c{c}\~ao no EPOS no segundo semestre e fazer testes de integra\c{c}\~ao.

\section{Cronograma}
\begin{ganttchart}[
        vgrid, hgrid, x unit=3mm, today=11, bar/.append style={fill=green},
    ] {1}{21}
    \gantttitle {Trabalho de Conclus\~ao de Curso}{21} \\
    \gantttitlelist {3,...,12}{2}\\
    \ganttgroup {2013.1} {1}{10}\\
    \ganttbar[progress=100]{Estudar EPOS} {1}{4}\\
    \ganttbar[progress=100]{Estudar EPOSMoteII}{1}{4}\\
    \ganttbar[progress=100]{Estudar modem GPRS}{1}{4}\\
    \ganttbar[progress=100]{Levantar requisitos libcoap no EPOS} {3}{4} \\
    \ganttbar[progress=100]{Levantar requisitos do HW} {1}{6}\\
    \ganttbar[progress=60]{Implementa\c{c}\~ao do CoAP} {3}{10} \\
    \ganttbar[progress=100]{Testar modem GPRS} {6}{10} \\
    \ganttbar[progress=0] {Desenvolvimento do prot\'otipo do HW} {7}{10}\\
    \ganttmilestone {Relat\'orio de TCC I}{10}\\
    \ganttnewline[thick, blue]
    \ganttgroup {2013.2}{11}{20} \\
    \ganttbar[progress=20]{Implementa\c{c}\~ao do Gateway} {11}{18}\\
    \ganttbar[progress=0]{Testes de Integra\c{c}\~ao} {19}{20}\\
    \ganttbar[progress=40]{Documenta\c{c}\~ao do trabalho} {1}{20} \\
    \ganttmilestone {Relat\'orio de TCC II}{20}

    \ganttlink {elem3}{elem7}
    \ganttlink {elem5}{elem8}
    \ganttlink {elem7}{elem11}
    \ganttlink {elem8}{elem12}
    \ganttlink {elem11}{elem12}

\end{ganttchart}

\section{Resultados parciais}

Abaixo est\~ao listadas as atividades conclu\'idas at\'e o momento:
\begin{itemize}
    \item Estudos realizados durante o desenvolvimento do protocolo e testes para valida\c{c}\~ao do m\'odulo GPRS. e durante o levamento de requisitos para portar a libcoap para o EPOS
    \item Estudar EPOSMoteII.
    \item Levantar requisitos libcoap no EPOS.
    \item Levantar requisitos do HW.
    \item Implementa\c{c}\~ao do CoAP: a parte de valida\c{c}\~ao de um pacote CoAP foi feita com TDD para facilitar o verifica\c{c}\~ao de pacotes. Os testes para validar o envio/recebimento de mensagens do EPOS foi feito no qemu.
    \begin{itemize}[noitemsep,topsep=0pt,parsep=0pt,partopsep=0pt]
        \item Repassar mensagem para controle de Requisi\c{c}\~ao e Reposta.
        \item Adicionar a lista de mensagem recebidas.
        \item Envio de mensagens confirm\'aveis e n\~ao confirm\'aveis
        \item Mecanismo de retransmiss\~ao simplificado: enviar mensagens com confirma\c{c}\~ao pendentes e reconfigurar-se a pr\'oximo envdio.
    \end{itemize}
    
    \item Testes no m\'ouldo GPRS:
        \begin{itemize}[noitemsep,topsep=0pt,parsep=0pt,partopsep=0pt]
            \item Enviar Mensagem
            \item Receber Mensagem
            \item Criar socket TCP
            \item Enviar mensagem via socket
            \item Receber mensagem via socket
            \item Fazer requisi\c{c}\~ao HTTP para um webserver, foi poss\'ivel utilizando os comandos propriet\'arios do modem.
        \end{itemize}
    \item Desenvolvimento do prot\'otipo do HW.
\end{itemize}

\subsection{Atividades pendentes}
As atividades que ser\~ao feitas ao longo do segundo semestre:
\begin{itemize}
    \item Implementa\c{c}\~ao do Gateway.
    \item Testes de Integra\c{c}\~ao.
    \item Documenta\c{c}\~ao do trabalho.
    \item Relat\'orio de TCC II.
\end{itemize}
