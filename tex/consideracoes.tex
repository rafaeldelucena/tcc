\subsection{Conclus\~ao}

O objetivo principal foi atigindo, o desenvolvimento de um gateway simplificado para redes de sensores que utilizem o protocolo CoAP para disponibilizar servi\c{c}os de sensoriamente e atua\c{c}\~ao.
Al\'em disso o consumo de energia e o espa\c{c}os de armazenamento s\~ao muito baixos, respeitando os requis\'itos de um n\'o sensor.

\afazer{em desenvolvimento...}

\subsection{Trabalhos Futuros}

Entres os trabalhos futuros poss\'iveis se destacam os seguintes:
A implementa\c{c}\~ao do protocolo coaps, utilizando DLTS para comunica\c{c}\~ao segura entre os n\'os.

Uma implementa\c{c}\~ao de servidor CoAP completa para executar utilizando a plataforma do EPOSMote II.

Implementa\c{c}\~ao de um gateway que utilize Software Defined Radio, assim apenas um transceiver ser\'a necess\'ario para
fazer a integra\c{c}\~ao com a Internet.

Um gerador de c\'odigo que utiliza como entrada uma linguagem de especifica\c{c}\~ao dos poss\'iveis recursos e como sa\'ida c\'odigo ANSI C m\'inimo de um servidor web utiliza CoAP e seus respectivos recursos. Este gerador deve ser gen\'erico suficiente para ser f\'acil a adapta\c{c}\~ao de diferentes pilhas UDP/IP, arquiteturas e tipos de sensores.


\afazer{em desenvolvimento...}
