Este trabalho apresentou alguns conceitos sobre redes de sensores e atuadores que utilizam um protocolo de aplica\c{c}\~ao espec\'ifico, para disponibilizar recursos diversos e facilitar a interoperabilidade de diversos sistemas.

As principal contribui\c{c}\~oes deste trabalho s\~ao a avalia\c{c}\~ao de diferentes implementa\c{c}\~oes em diversas plataformas e a implementa\c{c}\~ao de uma infraestrutura para o desenvolvimento de aplica\c{c}\~oes que utilizem redes de sensores sem fio.

O objetivo principal foi atigindo, o desenvolvimento de um gateway simplificado para redes de sensores que repassa as mensagens recebidas para um servidor CoAP na Internet. \'E possui acessar o sistema utilizando um cliente HTTP, facilitando a disponibiliza\c{c}\~ao de servi\c{c}os de sensoriamente e atua\c{c}\~ao.

O protocolo CoAP \'e adequado a redes com um elevado n\'umero de n\'os baseando-se na arquitetura da WEB, focando na utiliza\c{c}\~ao dos princ\'ios REST para o desenvolvimento de novas aplica\c{c}\~oes.

O tamanho do c\'odigo respeitou o espa\c{c}o de armazenamento restrito, comum desse tipo de aplica\c{c}\~ao e foi poss\'ivel criar uma aplica\c{c}\~ao que possa ser utilizado em uma vasta gama de dispositivos. Entres os trabalhos futuros poss\'iveis se destacam os seguintes:

Uma implementa\c{c}\~ao de servidor CoAP completa para executar utilizando a plataforma do EPOSMote II. Para isto seria necess\'ario desenvolver um m\'odulo gen\'erico o suficiente para adicionar diversos recursos como sensores e atuadores, com suporte a diferentes tipos de fun\c{c}\~oes, dados e a\c{c}\~oes como coletar dado de um ADC ou enviar comandos a um servo.

A implementa\c{c}\~ao da vers\~ao segura do protocolo CoAP, utilizando DLTS para comunica\c{c}\~ao segura entre os n\'os. Melhores formas de medir a performance da solu\c{c}\~ao protosta como: utiliza\c{c}\~ao da mem\'oria ao longo do tempo, vaz\~ao da troca de mensagens e RTT.

Uma aplica\c{c}\~ao que poderia reduzir ainda mais o custo para integrar as redes a Internet, \'e a implementa\c{c}\~ao de um gateway que utilize Software Defined Transceiver, assim \'e poss\'ivel integrar uma gama enorme de dispositivos e utilizar apenas um m\'odulo transceiver para enviar os dados da RSSF para Internet.

Um gerador de c\'odigo que utiliza como entrada uma linguagem de especifica\c{c}\~ao dos poss\'iveis recursos e como sa\'ida c\'odigo ANSI C m\'inimo de um servidor web utiliza CoAP e seus respectivos recursos. Este gerador deve ser gen\'erico suficiente para ser f\'acil a adapta\c{c}\~ao de diferentes pilhas CoAP/UDP/IP, arquiteturas e tipos de sensores.

Outra trabalho futuro interessante seria o desenvolvimento de um modelo para apresenta\c{c}\~ao dos dados coletados dos recursos dispon\'iveis na rede.
