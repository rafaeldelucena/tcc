\section{An\'alise Funcional}

A aplica\c{c}\~ao gateway \'e respons\'avel por fazer uma ponte entre a rede de sensores e a Internet. \'E poss\'ivel descobrir servi\c{c}os e recursos, iniciar requisi\c{c}\~oes, fazer descoberta de recursos, tudo  isto \'e transparente para os pontos comunicantes. 

Atualmente a aplica\c{c}\~ao n\~ao possui uma implementa\c{c}\~ao segura do procotolo CoAP, que utiliza o DTLS. Para resolver este problema \'e necess\'ario que o pacote seja criptografado.

\section{An\'alise Estrutural}

Um Fator muito importante para verificar a implemenca\c{c}\~ao de protocolo \'e a interoperabilidade entre diferentes implementa\c{c}\~oes. \'E essencial que a implementa\c{c}\~ao cliente trate os mecanismos de requisi\c{c}\~ao e resposta tanto para mensagens confirmadas e n\~ao confirmadas. Uma maneira de validar a interoperabilidade da implementa\c{c}\~ao do protocolo \'e usar um padr\~ao de testes entre diferentes sistemas.

A ETSI em conjunto com a IPSO desenvolveram um conjunto de testes para validar o comportamento entre diversas implementa\c{c}\~oes CoAP. Este teste foi aplicado em 24, 25 de mar\c{c}o de 2012 em Paris, conhecido como primeiro evento IOT CoAP plugtest. Para validar a interoperabilidade os teste s\~ao: especifica\c{c}\~ao b\'asica do CoAP, transfer\^encia em bloco, observa\c{c}\~ao de recursos CoAP, formato CORE link. Os teste s\~ao executados entre diferentes dispositivos e implementa\c{c}\~oes CoAP. O cen\'ario inicial de testes possui os seguintes requisitos:
\begin{itemize}
    \item Cada equipamento deve estar configurado com um endere\c{c}o unicast.
    \item A cache do cliente deve estar limpa.
    \item Utiliza\c{c}\~ao da op\c{c}\~ao ETag por padr\~ao deve ser evitada a n\~ao ser que esteja explicitamente descrito no caso de teste. 
    \item O uso de Tokens deve ser evitado a n\~ao ser que o caso de teste utilize, por\'em a implementa\c{c}\~ao deve estar preparada para tratar o token.
    \item O uso de repostas por Piggybacked deve ser preferencial, a menos que a descri\c{c}\~ao do teste altere este padr\~ao.
\end{itemize}

A tabela abaixo \ref{plugTest} apresenta a rela\c{c}\~ao dos testes, na coluna resultado o s\'imbolo "\xmark" representa falha e "\cmark" sucesso nos testes realizados para a solu\c{c}\~ao desenvolvida neste trabalho. O teste consistiu em executar localmente os diferentes servidores e o cliente desenvolvido.

\begin{table}[H]
\centering
\label{plugTest}
\begin{tabular}{p{7cm}|c}
\hline
\multicolumn{2}{c}{\bfseries{Testes para especifica\c{c}\~ao b\'asica CoAP}} \\ \hline
\rowcolor[HTML]{ECF4FF}
\multicolumn{1}{c|}{Cen\'ario} & \multicolumn{1}{c}{Resultado} \\ \hline
Tratar GET, confirm\'avel. & \cmark \\
Tratar POST, confirm\'avel. & \cmark \\
Tratar PUT, confirm\'avel. & \cmark \\
Tratar DELETE, confirm\'avel. & \cmark \\
Tratar GET, sem confirma\c{c}\~ao. & \cmark \\
Tratar POST, sem confirma\c{c}\~ao. & \cmark \\
Tratar PUT, sem confirma\c{c}\~ao. & \cmark \\
Tratar DELETE, sem confirma\c{c}\~ao. & \cmark \\
Tratar GET com resposta separada. & \cmark \\
Tratar requisi\c{c}\~ao com Token. & \cmark \\
Tratar requisi\c{c}\~ao sem Token. & \cmark \\
Tratar requisi\c{c}\~ao op\c{c}\~oes URI-Path. & \cmark \\
Tratar requisi\c{c}\~ao op\c{c}\~oes URI-Query. & \cmark \\
Interoperablidade em contexto de perda\\(CON mode, piggybacked response) & \cmark \\
Interoperablidade em contexto de perda\\(CON mode, delayed response) & \cmark \\
Tratar GET com resposta separada, sem confirma\c{c}\~ao. & \cmark \\ \hline
\multicolumn{2}{c}{\bfseries{Testes para validar o formato de dados CORE link Format}} \\ \hline
\rowcolor[HTML]{ECF4FF}
\multicolumn{1}{c|}{Cen\'ario} & \multicolumn{1}{c}{Resultado} \\ \hline
Descoberta de recursos well-known. & \xmark \\
Utiliza\c{c}\~ao de consulta para filtrar resultados. & \xmark \\ \hline
\multicolumn{2}{c}{\bfseries{Testes para validar a transfer\^encia de blocos}}\\ \hline
\rowcolor[HTML]{ECF4FF}
\multicolumn{1}{c|}{Cen\'ario} & \multicolumn{1}{c}{Resultado} \\ \hline
Transfer\^encia de blocos grandes utilizando GET (negocia\c{c}\~ao antecipada). & \xmark \\
Transfer\^encia de blocos grandes utilizando GET (negocia\c{c}\~ao atrasada). & \xmark \\
Transfer\^encia de blocos grandes utilizando o PUT. & \xmark \\
Transfer\^encia de blocos grandes utilizando o POST. & \xmark \\ \hline
\multicolumn{2}{c}{\bfseries{Testes para observa\c{c}\~ao de recursos}} \\ \hline
\rowcolor[HTML]{ECF4FF}
\multicolumn{1}{c|}{Cen\'ario} & \multicolumn{1}{c}{Resultado} \\ \hline
Tratar observa\c{c}\~ao de recursos. & \xmark \\
Parar a observa\c{c}\~ao de recursos. & \xmark \\
Detec\c{c}\~ao de deregistro do cliente (Max-Age). & \xmark \\
Detec\c{c}\~ao de deregistro do servidor (client OFF). & \xmark \\
Detec\c{c}\~ao de deregistro do servidor (RESET expl\'icito). & \xmark \\ \hline
\end{tabular}
\caption{Resultados dos testes de interoperabilidade IOT Plugtest.}
\end{table}

A implementa\c{c}\~ao ir\'a utilizar apenas pacotes com no m\'aximo de 128 bytes, portanto o suporte a transfer\^encia de grandes blocos fragmentados n\~ao ser\'a necess\'ario no momento.

O gateway utiliza apenas uma implementa\c{c}\~ao cliente CoAP, n\~ao sendo necess\'ario possuir a implementa\c{c}\~ao de descoberta de recursos e observa\c{c}\~ao, pois estas funcionalidade est\~ao implementada nos nodos sensores. O gateway ir\'a repassar a requisi\c{c}\~ao para os webservers e a cada resposta encaminhar\'a para a Internet, simplificando bastante a implementa\c{c}\~ao. A id\'eia \'e o gateway ser totalmente transparente tanto para clientes e servidores.

Para a aplica\c{c}\~ao proposta o conjunto de funcionalidades m\'inimas passou no teste de interoperabilidade, \'e necess\'ario medir a performance, a plataforma utilizada possu\'i diversas restri\c{c}\~oes j\'a citadas.

\section{An\'alise de Desempenho}

Afim de mensurar algumas informa\c{c}\~oes importantes no contexto da aplica\c{c}\~ao foi utilizado a vers\~ao 4.8.1 do compilador GCC para arquitetura x86. Para a aplica\c{c}\~ao no EPOS,  a vers\~ao do compilador GCC \'e a 4.4, dispon\'ivel em \cite{eposProject}.

As flags de compila\c{c}\~ao utilizadas foram as padr\~oes de cada sistema de constru\c{c}\~ao dos projetos para a plataforma x86. Abaixo a tabela \ref{comparacaoCoap} mostra um comparativo do espa\c{c}o utilizado das diferentes implementa\c{c}\~oes CoAP.

\begin{table}[H]
\label{comparacaoCoap}
\centering
\begin{tabular}{@{}ccccccc@{}}
    \toprule
    OS & CoAP & App & .text & .data & .bss & Total\\ \midrule
    Contiki& Erbium & coap-client-example & 97995 & 1792 & 16864 & 116651 \\
    EPOS & CantCoap & CoapClient & 52348 & 117 & 9652 & 62117 \\
    Linux & libCoap & coap-client & 55194 & 1346 & 2056 & 58596 \\
    \bottomrule
\end{tabular}
\caption{Compara\c{c}\~ao das implementa\c{c}\~oes em consumo de mem\'oria em bytes}
\end{table}

Observa-se que a solu\c{c}\~ao do EPOS \'e no total de mem\'oria utilizada \'e cerca de 46\% menor que a implementa\c{c}\~ao do Contiki e 7\% maior que a implementa\c{c}\~ao utilizando a libCoap. Apesar de n\~ao ser a mais compacta \'e suficientemente pequena para ser utilizada na aplica\c{c}\~ao.

A grande quantidade de dados inicializados na implementa\c{c}\~ao deve-se ao fato do uso de buffers de 128 bytes para cada pacote, afim de simplificar a ger\^encia de mem\'oria do software desenvolvido. %Em aplica\c{c}\~oes restritas que n\~ao possuem unidades de ger\^encia de mem\'oria isto \'e interessante

%O tempo de execu\c{c}\~ao das principais fun\c{c}\~oes tamb\'em foi medido. Abaixo a tabela \ref{executionTimeCoap}, que faz um comparativo com as diversas arquituras e opera\c{c}\~oes entres as principais solu\c{c}\~oes livres.
%
%\begin{table}[H]
%    \label{executionTimeCoap}
%    \centering
%    \begin{tabular}{@{}ccc@{}}
%        \toprule
%        CoAP & Function & Time in miliseconds \\ \midrule
%        Erbium & coap\textunderscore parser &  \\
%        libCoap & coap\textunderscore parse \textunderscore pdu & \\
%        CantCoap & CoapPDU & \\
%        \bottomrule
%    \end{tabular}
%    \caption{Compara\c{c}\~ao das implementa\c{c}\~oes em tempo de execu\c{c}\~ao}
%\end{table}
%Tabela que demonstra a vaz\~ao do sistema para mensagens, com tamanho definido anteriormente.

%Abaixo um comparativo entre as principais fun\c{c}\~oes utilizadas nos sistema, utilizando um pacote padr\~ao e um pacote com tamanho m\'aximo de 128 bytes, o valor m\'aximo de MTU utilizado.
%
%\begin{table}[H]
%    \label{throughputCoap}
%    \centering
%        \begin{tabular}{@{}cccc@{}}
%            \toprule
%            OS & CoAP & App & N\'umero de Mensagens\\ \midrule
%            Contiki & Erbium & ER-Rest-Server &  \\
%            Linux & libCoap & Coap Server &  \\
%            OpenOS & OpenCoap &  &  \\
%            \bottomrule
%        \end{tabular}
%    \caption{Compara\c{c}\~ao da vaz\~ao entre as  implementa\c{c}\~oes}
%\end{table}
