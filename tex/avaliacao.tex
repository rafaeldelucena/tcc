\section{An\'alise Funcional}
\subsection{Limita\c{c}\~oes Funcionais}
A falta de uma implementaca\c{c}\~ao segura, utilizando os padr\~oes DTLS.
\section{An\'alise Estrutural}
\lstset{ %
  backgroundcolor=\color{white},   % choose the background color; you must add \usepackage{color} or \usepackage{xcolor}
  basicstyle=\footnotesize,        % the size of the fonts that are used for the code
  breakatwhitespace=false,         % sets if automatic breaks should only happen at whitespace
  breaklines=true,                 % sets automatic line breaking
  extendedchars=true,              % lets you use non-ASCII characters; for 8-bits encodings only, does not work with UTF-8
}
A biblioteca utilizada para montar o pacote CoAP foi:
\begin{lstlisting}
    https://github.com/staropram/cantcoap.git
 \end{lstlisting} Na qual enviei algumas corre\c{c}\~oes e testes para facilitar a verifica\c{c}\~ao da execu\c{c}\~ao correta dos algoritmos internos durante mudan\c{c}as no c\'odigo. As altera\c{c}\~oes podem ser visualisadas aqui: \begin{lstlisting}
    https://github.com/staropram/cantcoap/commits?author=rafaeldelucena.
\end{lstlisting}

Para o funcionamento desta biblioteca no EPOS, e para utilizar uma MTU limitada a 128 bytes utilizo um buffer com um valor m\'aximo e armazeno os dados do pacote no buffer. Foi necess\'ario alterar os tipos das vari\'aveis para se adquerem ao EPOS.

O desenvolvimento de um mecanismo de retransmiss\~ao de mensagens n\~ao-confirmadas utilizando uma lista ordenada. Mecanismo de requisic\c{c}\~o e resposta, as requisi\c{c}\~oes pendentes foram armezenadas num Hash com a chave sendo o token gerado pelo cliente.

Para validar o comportamento utilizei alguns testes da pr\'opria biblioteca CoAP portados para o EPOS. Foi necess\'ario implementar a func\c{c}\~ao assert.

\section{An\'alise de Desempenho}

O tempo de execu\c{c}\~ao das principais fun\c{c}\~oes foi medido. Abaixo uma tabela comparativa com as diversas arquituras e opera\c{c}\~oes entres as principais solu\c{c}\~oes livres.

\begin{table}[h]
\begin{tabular}{@{}lllll@{}}
\toprule
OS & CoAP & Size &  Memory &  &  \\ \midrule
EPOS &  CantCoap &  &  &  \\
Contiki &  Erbium &  &  &  \\
Contiki &  libcoap &  &  &  \\
TinyOS &  libcoap &  &  &  \\ \bottomrule
\end{tabular}
\end{table}

%\lstdefinestyle{customc}{
%  belowcaptionskip=1\baselineskip,
%  breaklines=true,
%  xleftmargin=\parindent,
%  language=C,
%  showstringspaces=false,
%  basicstyle=\footnotesize\ttfamily,
%  keywordstyle=\bfseries\color{green!40!black},
%  commentstyle=\itshape\color{purple!40!black},
%  identifierstyle=\color{blue},
%  stringstyle=\color{orange},
%}
%
%\lstdefinestyle{customasm}{
%  belowcaptionskip=1\baselineskip,
%  frame=L,
%  xleftmargin=\parindent,
%  language=[x86masm]Assembler,
%  basicstyle=\footnotesize\ttfamily,
%  commentstyle=\itshape\color{purple!40!black},
%}
%
%\lstset{escapechar=@,style=customc}
%
%\begin{lstlisting}
%#include <stdio.h>
%#define N 10
%/* Block
% * comment */
% 
%int main()
%{
%    int i;
% 
%    // Line comment.
%    puts("Hello world!");
% 
%    for (i = 0; i < N; i++)
%    {
%        puts("LaTeX is also great for programmers!");
%    }
% 
%    return 0;
%}
%\end{lstlisting}
