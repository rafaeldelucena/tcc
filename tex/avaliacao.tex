\section{An\'alise Funcional}

A aplica\c{c}\~ao gateway \'e respons\'avel por fazer uma ponte entre a rede de sensores e a Internet. \'E poss\'ivel descobrir servi\c{c}os e recursos, iniciar requisi\c{c}\~oes, fazer descoberta de recursos, tudo  isto \'e transparente para os pontos comunicantes. 

Atualmente a aplica\c{c}\~ao n\~ao possui uma implementa\c{c}\~ao segura do procotolo CoAP, que utiliza o DTLS. Para resolver este problema \'e necess\'ario que o pacote seja criptografado.

\section{An\'alise Estrutural}

Um Fator muito importante para verificar a implemenca\c{c}\~ao de protocolo \'e a interoperabilidade entre diferentes implementa\c{c}\~oes. \'E essencial que a implementa\c{c}\~ao cliente trate os mecanismos de requisi\c{c}\~ao e resposta tanto para mensagens confirmadas e n\~ao confirmadas. Uma maneira de validar a interoperabilidade da implementa\c{c}\~ao do protocolo \'e usar um padr\~ao de testes entre diferentes sistemas.

A ETSI em conjunto com a IPSO desenvolveram um conjunto de testes para validar o comportamento entre diversas implementa\c{c}\~oes CoAP. Este teste foi aplicado em 24, 25 de mar\c{c}o de 2012 em Paris, conhecido como primeiro evento IOT CoAP plugtest. Para validar a interoperabilidade os teste s\~ao: especifica\c{c}\~ao b\'asica do CoAP, transfer\^encia em bloco, observa\c{c}\~ao de recursos CoAP, formato CORE link. Os teste s\~ao executados entre diferentes dispositivos e implementa\c{c}\~oes CoAP. O cen\'ario inicial de testes possui os seguintes requisitos:
\begin{itemize}
    \item Cada equipamento deve estar configurado com um endere\c{c}o unicast.
    \item A cache do cliente deve estar limpa.
    \item Utiliza\c{c}\~ao da op\c{c}\~ao ETag por padr\~ao deve ser evitada a n\~ao ser que esteja explicitamente descrito no caso de teste. 
    \item O uso de Tokens deve ser evitado a n\~ao ser que o caso de teste utilize, por\'em a implementa\c{c}\~ao deve estar preparada para tratar o token.
    \item O uso de repostas por Piggybacked deve ser preferencial, a menos que a descri\c{c}\~ao do teste altere este padr\~ao.
\end{itemize}

A tabela abaixo \ref{plugTest} mostra o resultado dos testes, os s\'imbolos "\xmark" e "\cmark" para a solu\c{c}\~ao desenvolvida neste trabalho.

\begin{table}[H]
\centering
\label{plugTest}
\begin{tabular}{p{8cm}|c{1cm}}
\hline
\rowcolor[HTML]{ECF4FF}
\multicolumn{1}{c|}{\textbf{CEN\'ARIO}} & \multicolumn{1}{c}{\textbf{RESULTADO}} \\ \hline
\multicolumn{2}{c}{\bfseries{Testes para especifica\c{c}\~ao b\'asica CoAP}} \\ \hline
Tratar GET, confirm\'avel. & \cmark \\
Tratar POST, confirm\'avel. & \cmark \\
Tratar PUT, confirm\'avel. & \cmark \\
Tratar DELETE, confirm\'avel. & \cmark \\
Tratar GET, sem confirma\c{c}\~ao. & \cmark \\
Tratar POST, sem confirma\c{c}\~ao. & \cmark \\
Tratar PUT, sem confirma\c{c}\~ao. & \cmark \\
Tratar DELETE, sem confirma\c{c}\~ao. & \cmark \\
Tratar GET com resposta separada. & \cmark \\
Tratar requisi\c{c}\~ao com Token. & \cmark \\
Tratar requisi\c{c}\~ao sem Token. & \cmark \\
Tratar requisi\c{c}\~ao op\c{c}\~oes URI-Path. & \cmark \\
Tratar requisi\c{c}\~ao op\c{c}\~oes URI-Query. & \cmark \\
Interoperablidade em contexto de perda\\(CON mode, piggybacked response) & \cmark \\
Interoperablidade em contexto de perda\\(CON mode, delayed response) & \cmark \\
Tratar GET com resposta separada, sem confirma\c{c}\~ao. & \cmark \\ \hline
\multicolumn{2}{c}{\bfseries{Testes para validar o formato de dados CORE link Format}} \\ \hline
Descoberta de recursos well-known. & \xmark \\
Utiliza\c{c}\~ao de consulta para filtrar resultados. & \xmark \\ \hline
\multicolumn{2}{c}{\bfseries{Testes para validar a transfer\^encia de blocos}}\\ \hline
Transfer\^encia de blocos grandes utilizando GET (negocia\c{c}\~ao antecipada). & \xmark \\
Transfer\^encia de blocos grandes utilizando GET (negocia\c{c}\~ao atrasada). & \xmark \\
Transfer\^encia de blocos grandes utilizando o PUT. & \xmark \\
Transfer\^encia de blocos grandes utilizando o POST. & \xmark \\ \hline
\multicolumn{2}{c}{\bfseries{Testes para observa\c{c}\~ao de recursos}} \\ \hline
Tratar observa\c{c}\~ao de recursos. & \xmark \\
Parar a observa\c{c}\~ao de recursos. & \xmark \\
Detec\c{c}\~ao de deregistro do cliente (Max-Age). & \xmark \\
Detec\c{c}\~ao de deregistro do servidor (client OFF). & \xmark \\
Detec\c{c}\~ao de deregistro do servidor (RESET expl\'icito). & \xmark \\ \hline
\end{tabular}
\caption{Resultados dos testes de interoperabilidade IOT Plugtest.}
\end{table}

O teste consistiu em executar localmente os diferentes servidores e o cliente desenvolvido. A implementa\c{c}\~ao ir\'a utilizar apenas pacotes com no m\'aximo de 128 bytes, portanto o suporte a transfer\^encia de grandes blocos fragmentados n\~ao ser\'a necess\'ario no momento.

O gateway utiliza apenas uma implementa\c{c}\~ao cliente CoAP, n\~ao sendo necess\'ario possuir a implementa\c{c}\~ao de descoberta de recursos e observa\c{c}\~ao, pois estas funcionalidade est\~ao implementada nos nodos sensores.

O gateway ir\'a repassar a requisi\c{c}\~ao para os webservers e a cada resposta encaminhar\'a para a Internet, simplificando bastante a implementa\c{c}\~ao. A id\'eia \'e o gateway ser totalmente transparente tanto para clientes e servidores. Para a aplica\c{c}\~ao proposta o conjunto de funcionalidades m\'inimas passou no teste de interoperabilidade.

\section{An\'alise de Desempenho}

A fim de mensurar algumas informa\c{c}\~oes importantes no contexto da aplica\c{c}\~ao foi utilizada a vers\~ao 4.4.5 para o GCC x86 e o GCC 4.3.2 (Sourcery G++ Lite 2008q3-66) para ARM no Contiki encontrada \cite{malvira}. A aplica\c{c}\~ao no EPOS utiliza a vers\~ao 4.4 do compilador GCC dispon\'ivel em \cite{eposProject}.

As flags de compila\c{c}\~ao utilizadas foram as padr\~oes de cada sistema de constru\c{c}\~ao dos projetos para a arquitetura x86 e para para ARM7 do MC1224V utilizado no EPOSMoteII e no Econotag, plataforma presente no Contiki e utilizado para comparativos. Abaixo a tabela \ref{comparacaoCoap} mostra um comparativo do espa\c{c}o utilizado das diferentes implementa\c{c}\~oes CoAP.

\begin{table}[H]
\label{comparacaoCoap}
\centering
\begin{tabular}{@{}p{1cm}p{1cm}p{1cm}r{1cm}r{1cm}r{1cm}r{1cm}@{}}
    \toprule
    ISA & OS & CoAP & .text & .data & .bss & Total\\ \midrule
    x86 32& Contiki & Erbium & 97995 & 1792 & 16864 & 116651 \\
    x86 32& EPOS & Cantcoap & 52348 & 117 & 9652 & 62117 \\
    x86 32& GNU Linux & libcoap & 45515 & 680 & 1952 & 48147\\ \hline
    ARM7 & Contiki & Erbium & 62727 & 440 & 18108 & 81275\\
    ARM7 & Contiki & Minimal Net & 59496 & 2432 & 22040 & 83968\\
    ARM7 & EPOS & Cantcoap & 62480 & 216 & 10504 & 73200\\
    \bottomrule
\end{tabular}
\caption{Compara\c{c}\~ao das implementa\c{c}\~oes em consumo de mem\'oria em bytes}
\end{table}

Nesta tabela \'e poss\'ivel verificar que para a arquitetura x86 a implementa\c{c}\~ao do trabalho \'e 30\% maior que a libcoap, por\'em em torno de 25\% menor que as implementa\c{c}\~oes do Contiki. Um dos fatores da diferen\c{c}a com a aplica\c{c}\~ao da libcoap \'e a utiliza\c{c}\~ao de bibliotecas din\^amicas \cite{ibmCodeSize}. Apesar de n\~ao ser a mais compacta \'e suficientemente pequena para ser utilizada na aplica\c{c}\~ao.

No gr\'afico \ref{memoryDiff} observa-se que a solu\c{c}\~ao do EPOS \'e no total de mem\'oria utilizada \'e cerca de 12,82\% menor que a implementa\c{c}\~ao do Contiki Minimal Net e 9,93\% menor que a implementa\c{c}\~ao utilizando o Erbium.

\begin{tikzpicture}
\centering
\label{memoryDiff}
\begin{axis}[ybar , enlargelimits =0.15, legend style ={ at ={(0.5,-0.15)}, anchor =north, legend columns =-1}, ylabel ={\# size in bytes}, symbolic x coords ={epos/cantcoap,contiki/minimal-net,contiki/erbium}, xtick =data, nodes near coords , nodes near coords align ={vertical},]
\addplot coordinates {(epos/cantcoap,62480) (contiki/minimal-net,59496) (contiki/erbium,62427)};
\addplot coordinates {(epos/cantcoap,216) (contiki/minimal-net,2432) (contiki/erbium,440)};
\addplot coordinates {(epos/cantcoap,10504) (contiki/minimal-net,22040) (contiki/erbium,18108)};
\addplot coordinates {(epos/cantcoap,73200) (contiki/minimal-net,83968) (contiki/erbium,81275)};
\legend {.text,.data,.bss, Total}
\end{axis}
\caption{Gr\'afico do tamanho das sess\~oes do arquivo ELF gerado para ARM.}
\end{tikzpicture}


Abaixo \ref{x86Elf} mais informa\c{c}\~oes do arquivo gerado ap\'os o sistema de constru\c{c}\~ao. Para compara\c{c}\~ao os s\'imbolos de depura\c{c}\~ao foram removidos do arquivo ELF. Para compara\c{c}\~ao 

\begin{lstlisting}[label={x86Elf},caption=Verificando o tipo de arquivo gerado no Linux utilizando a ferramenta file.]
file epos/img/coap_client
img/coap_client: ELF 32-bit LSB executable, Intel 80386, version 1 (SYSV), statically linked, stripped

file libcoap/examples/coap-client
examples/coap-client: ELF 32-bit LSB executable, Intel 80386, version 1 (SYSV), dynamically linked (uses shared libs), for GNU/Linux 2.6.18, stripped

file contiki/examples/er-rest-example/er-example-client.elf
er-example-client.elf: ELF 32-bit LSB executable, ARM, version 1 (SYSV), statically linked, stripped

file contiki/examples/rest-example/coap-client-example.elf 
coap-client-example.elf: ELF 32-bit LSB executable, ARM, version 1 (SYSV), statically linked, stripped

file epos/img/coap_client.img 
coap_client.img: ELF 32-bit LSB  executable, ARM, version 1, statically linked, stripped
\end{lstlisting}



A grande quantidade de dados inicializados na implementa\c{c}\~ao deve-se ao fato do uso de buffers de 128 bytes para cada pacote, afim de simplificar a ger\^encia de mem\'oria do software desenvolvido. %Em aplica\c{c}\~oes restritas que n\~ao possuem unidades de ger\^encia de mem\'oria isto \'e interessante

%O tempo de execu\c{c}\~ao das principais fun\c{c}\~oes tamb\'em foi medido. Abaixo a tabela \ref{executionTimeCoap}, que faz um comparativo com as diversas arquituras e opera\c{c}\~oes entres as principais solu\c{c}\~oes livres.
%
%\begin{table}[H]
%    \label{executionTimeCoap}
%    \centering
%    \begin{tabular}{@{}ccc@{}}
%        \toprule
%        CoAP & Function & Time in miliseconds \\ \midrule
%        Erbium & coap\textunderscore parser &  \\
%        libCoap & coap\textunderscore parse \textunderscore pdu & \\
%        CantCoap & CoapPDU & \\
%        \bottomrule
%    \end{tabular}
%    \caption{Compara\c{c}\~ao das implementa\c{c}\~oes em tempo de execu\c{c}\~ao}
%\end{table}
%Tabela que demonstra a vaz\~ao do sistema para mensagens, com tamanho definido anteriormente.

%Abaixo um comparativo entre as principais fun\c{c}\~oes utilizadas nos sistema, utilizando um pacote padr\~ao e um pacote com tamanho m\'aximo de 128 bytes, o valor m\'aximo de MTU utilizado.
%
%\begin{table}[H]
%    \label{throughputCoap}
%    \centering
%        \begin{tabular}{@{}cccc@{}}
%            \toprule
%            OS & CoAP & App & N\'umero de Mensagens\\ \midrule
%            Contiki & Erbium & ER-Rest-Server &  \\
%            Linux & libCoap & Coap Server &  \\
%            OpenOS & OpenCoap &  &  \\
%            \bottomrule
%        \end{tabular}
%    \caption{Compara\c{c}\~ao da vaz\~ao entre as  implementa\c{c}\~oes}
%\end{table}
