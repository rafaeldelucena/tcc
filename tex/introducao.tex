Redes de sensores e atuadores s\~ao utilizadas para a capta\c{c}\~ao, processamento de informa\c{c}\~ao e atua\c{c}\~ao sobre um ambiente, tornando-as importantes em aplica\c{c}\~oes de controle, telemetria e rastreamento de sistemas.

N\'os que participam destas redes geralmente s\~ao compostos por computadores e r\'adios simplificados, que possuem restri\c{c}\~oes de mem\'oria, processamento, energia e capacidade de comunica\c{c}\~ao, mas um custo relativamente baixo de equipamentos.

O maior consumo de energia neste tipo de aplica\c{c}\~ao \'e do r\'adio, portanto o desafio dos algoritmos de comunica\c{c}\~ao nesta \'area \'e manter os r\'adios ligados o m\'inimo de tempo poss\'ivel sem comprometer a conectividade do n\'o.

\section{Objetivos}
O objetivo geral desse trabalho \'e descrever e implementar webservices em uma rede sensores sem fio que far\~ao a aquisi\c{c}\~ao dos dados do ambiente e disponibilizar\~ao as informa\c{c}\~oes captadas na Internet.

\subsection{Objetivos Espec\'ificos}
Este trabalho realizar\'a o desenvolvimento de aplica\c{c}\~oes web e software de sistema para fazer a ponte entre a rede de sensores e a Internet. O Sistema operacional utilizado ser\'a o EPOS, que possui uma implementa\c{c}\~ao de pilha UDP/IP. A aplica\c{c}\~ao integradora GPRS/802.15.4 ir\'a executar na plataforma EposMoteII utilizando uma extens\~ao GPRS.

A comunica\c{c}\~ao entre os n\'os da rede ser\'a feita atrav\'es do protocolo de aplica\c{c}\~ao CoAP, um protocolo espec\'ifico para redes de sensores sem fio. Ser\'a utilizado um porte de uma implementa\c{c}\~ao livre do protocolo CoAP.

Sendo assim, os objetivos espec\'ificos s\~ao:

\begin{enumerate}
    \item{Portar o protocolo CoAP para o EPOS;}
    \item{Implementar uma aplica\c{c}\~ao para redes de sensores sem fio;}
    \item{Desenvolver a aplica\c{c}\~ao gateway GPRS/802.15.4.}
    \item{Desenvolver uma aplica\c{c}\~ao web para visualiza\c{c}\~ao da informa\c{c}\~ao;}
    \item{Avaliar a solu\c{c}\~ao desenvolvida.}
\end{enumerate}

\section{Justificativa}

Os mecanismos de confiabilidade na transmiss\~ao de dados, t\'ecnicas para se manter uma conex\~ao do TCP e rearranjos que s\~ao feitos para garantir a ordem das mensagens recebidas n\~ao s\~ao adequados para um dispositivos com suprimento limitado de energia, como uma bateria ou uma placa fotovolt\'aica. Estas t\'ecnicas fazem que os transmissores fiquem ligados por mais tempo, para manter a conex\~ao ou at\'e mesmo para reenvio de mensagens.

O maior consumo de energia de um n\'o sensor \'e no envio e recebimento de dados, quando mantem seu transmissor ligado. Al\'em disso quem recebe a mensagem precisa mont\'a-la e tratar as partes corrompidas, podendo gerar retransmiss\~oes.

Por sua vez o protocolo do UDP, n\~ao mant\'em conex\~ao, dados s\~ao recebidos fora de ordem e o envio \'e feito de uma mensagem por vez. Isto implica tamb\'em na redu\c{c}\~ao do tamanho do cabe\c{c}alho do pacote.

Estas caracter\'isticas demostram uma alternativa interessante para estes equipamentos limitados. Testes feitos em implementa\c{c}\~oes de sistemas operacionais similares ao EPOS, como Contiki e TinyOS, utilizando o protocolo CoAP demonstram redu\c{c}\~ao no consumo de energia e mem\'oria em rela\c{c}\~ao ao HTTP \cite{kuladinithi2011implementation}.

A falta de padroniza\c{c}\~ao dos protocolos afeta o desenvolvimento de uma rede p\'ublica ub\'iqua de uma cidade inteligente, por exemplo. Grande parte das solu\c{c}\~oes utiliza protocolos propriet\'arios que se comunicacam apenas com os produtos de um mesmo fabricante.

O protocolo HTTP foi desenvolvido para comunica\c{c}\~ao de computadores de prop\'osito geral, onde as restri\c{c}\~oes citadas n\~ao s\~ao comuns. Em rela\c{c}\~ao ao tamanho, o pacote HTTP \'e um problema para redes 802.15.4, j\'a que estas redes possuem uma restri\c{c}\~ao de 128 bytes em sua PDU. O protocolo TCP precisa transmitir mensagens adicionais para manter uma conex\~ao, outra caracter\'istica que n\~ao \'e interessante para RSSF.

Um protocolo leve como CoAP pode tornar vi\'avel a cria\c{c}\~ao de aplica\-\c{c}\~oes web em redes de sensores sem fio por um baixo custo. Neste trabalho \'e proposto uma infraestrutura de comunica\c{c}\~ao entre redes de sensores sem fio e a Internet, utilizando protocolos leves entre os n\'os sensores e um gateway GPRS para \'areas sem acesso \`a WIFI, aproveitando a vasta abrang\^encia da tecnologia de telefonia. Com a utiliza\c{c}\~ao do CoAP \'e esperado uma redu\c{c}\~ao de consumo de energia e mem\'oria, em rela\c{c}\~ao a outros protocolos de aplica\c{c}\~ao existentes.

Em lugares aonde n\~ao existe o acesso a rede cabeada ou sem fio, como lugares afastados, na \'area rural, por exemplo a distribui\c{c}\~ao da informa\c{c}\~ao para Internet ser\'a feita atrav\'es de um gateway.

O gateway ser\'a composto por um EposMoteII e um m\'odulo GPRS, respons\'avel por fazer a ponte entre a rede de sensores e a Internet. Atualmente o padr\~ao GPRS oferece a maior cobertura dentre as tecnologias de transmiss\~ao de telefonia no Brasil, atingindo cerca de 5477 munic\'ipios.\cite{coberturaGPRS}


\section{Metodologia}

Ser\'a feito um levantamento dos componentes de software e hardware necess\'arios para o desenvolvimento do gateway 802.15.4/GPRS. Neste caso utilizando o mote EPOSMote II e um m\'odulo GPRS, que ser\'a desenvolvido em paralelo a implementa\c{c}\~ao do protocolo de aplica\c{c}\~ao CoAP no sistema operacional EPOS.

Durante o desenvolvimento do protocolo, testes ser\~ao executados para verificar o correto comportamento e diminuir a depura\c{c}\~ao em hardware, que geralmente leva mais tempo.

Nos testes de integra\c{c}\~ao do gateway, ser\'a utilizada uma placa de desenvolvimento em conjunto com um m\'odulo M95 da Quectel disponibilizada pelo LISHA. Ser\~ao realizads testes de envio de mensagens em diversos protocolos, inclusive testes com comandos propriet\'arios adicionais do modem.

Para testes de integra\c{c}\~ao, as aplica\c{c}\~oes ser\~ao executadas na plataforma de sensores sem fio EPOS Mote II utilizando o EPOS com o CoAP desenvolvido.
