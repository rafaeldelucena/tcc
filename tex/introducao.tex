Redes de sensores s\~ao utilizadas para a capta\c{c}\~ao, processamento de informa\c{c}\~ao e atua\c{c}\~ao sobre um ambiente, tornando-as importantes para controle, telemetria e rastreamento de sistemas.
Os n\'os que participam destas redes geralmente s\~ao computadores e r\'adios simplificados, que possuem restri\c{c}\~oes de mem\'oria, processamento, energia e comunica\c{c}\~ao, mas um custo relativamente baixo de equipamentos. O maior consumo de energia neste tipo de aplica\c{c}\~ao \'e o r\'adio, portanto os desafios dos algoritmos de roteamento nesta \'area s\~ao em manter os r\'adios o m\'inimo per\'iodo de tempo poss\'ivel e manter o n\'o comunic\'avel pela rede.

\section{OBJETIVOS}

Descri\c{c}\~ao...

\subsection{Objetivo Geral}

O trabalho ser\'a descrever e implementar webservices em uma rede sensores sem fio, que far\~ao a aquisi\c{c}\~ao dos dados do ambiente e disponibilizar\~ao as informa\c{c}\~oes captadas. A comunica\c{c}\~ao entre os n\'os sensores ser\'a feita utilizando um protocolo de aplica\c{c}\~ao padr\~ao, apropriado para redes de sensores sem fio, viabilisando o desenvolvimento de webservices para os n\'os da rede.
A distribui\c{c}\~ao da informa\c{c}\~ao para Internet ser\'a feita atrav\'es de um gateway. Os objetivos s\~ao: o monitoramento de ambientes e a integra\c{c}\~ao da rede de sensores sem fio com a Internet em lugares aonde o n\~ao existe o acesso a rede cabeada ou sem fio, como lugares afastados, na \'area rural, por exemplo. GPRS possui a maior cobertura dentre as tecnologias de transmiss\~ao de telefonia no Brasil, atingindo cerca de 5477 munic\'ipios.


\subsection{Objetivos Espec\'ificos}

Implementar uma aplica\c{c}\~ao de redes de sensores sem fio, que utilize um gateway GPRS/Zigbee que ser\'a respons\'avel em disponibilizar as informa\c{c}\~oes captadas pela rede para Internet, al\'em de oferecer uma simples aplica\c{c}\~ao web que ser\'a poss\'ivel para o usu\'ario configurar os par\^ametros da rede.
Na infra-estrutura de software ser\'a necess\'ario implementar o protocolo CoAP no EPOS. Desenvolver o protocolo CoAP utilizando a pilha UDP do EPOS ou portar a biblioteca libcoap, uma biblioteca de c\'odigo aberto desenvolvida em C que implementa o protocolo de aplica\c{c}ao leve para dispositivos com restri\c{c}\~oes de recursos como poder computacional, alcance de r\'adio, mem\'oria, vaz\~ao ou tamanho dos pacotes de rede.
O gateway 6LowPan/GPRS ter\'a a capacidade de fazer a ponte entre a rede de sensores e a Internet, utilizando um modem GPRS. Ser\'a projetado utilizando um mote, um modem GPRS al\'em do circuito necess\'ario para integra\c{c}\~ao dos dois m\'odulos. Ser\'a a parte de hardware do projeto.
