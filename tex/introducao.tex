Redes de sensores s\~ao utilizadas para a capta\c{c}\~ao, processamento de informa\c{c}\~ao e atua\c{c}\~ao sobre um ambiente, tornando-as importantes em aplica\c{c}\~oes de controle, telemetria e rastreamento de sistemas.

N\'os que participam destas redes geralmente s\~ao computadores e r\'adios simplificados, que possuem restri\c{c}\~oes de mem\'oria, processamento, energia e capacidade de comunica\c{c}\~ao, mas um custo relativamente baixo de equipamentos.

O maior consumo de energia neste tipo de aplica\c{c}\~ao \'e do r\'adio, portanto os desafios dos algoritmos de comunica\c{c}\~ao nesta \'area s\~ao manter os r\'adios ligados o m\'inimo de tempo poss\'ivel sem comprometer a conectividade do n\'o.

\section{Objetivos}
O objetivo geral desse trabalho \'e descrever e implementar webservices em uma rede sensores sem fio, que far\~ao\
a aquisi\c{c}\~ao dos dados do ambiente e disponibilizar\~ao as informa\c{c}\~oes captadas na Internet.

\subsection{Objetivos Espec\'ificos}
Os objetivos espec\'ificos do trabalho s\~ao relacionados ao desenvolvimento de duas aplica\c{c}\~oes e uma biblioteca.

%TODO Melhorar!

\begin{itemize}
    \item Portar o protocolo CoAP, Constrained Application Protocol para o EPOS, Embedded Paralell Operating System.
    \item Implementar uma aplica\c{c}\~ao de redes de sensores sem fio.
    \item Desenvolver o firmware do  gateway GPRS/Zigbee, que ir\'a disponibilizar as informa\c{c}\~oes captadas da rede para Internet.
\end{itemize}
        

\section{Justificativa}

Os mecanismos de confiabilidade na transmiss\~ao de dados, t\'ecnicas para se manter uma conex\~ao do TCP e rearranjos que s\~ao feitos para garantir a ordem das mensagens recebidas n\~ao s\~ao adequados para um dispositivos com suprimento limitado de energia, como uma bateria ou uma placa fotovolt\'aica. Esstas t\'ecnicas fazem que os transmissores fiquem ligados por mais tempo, para manter a conex\~ao ou at\'e mesmo para reenvio de mensagens. O maior consumo de energia de um n\'o sensor \'e no envio e recebimento de dados, quando mantem seu transmissor ligado. Al\'em quem recebe a mensagem precisa mont\'a-la e tratar as partes corrompidas.

Por sua vez o protocolo do UDP, n\~ao mant\'em conex\~ao, dados s\~ao recebidos fora de ordem e o envio \'e feito de uma mensagem por vez.  Isto implica tamb\'em na redu\c{c}\~ao do tamanho do cabe\c{c}alho do pacote.

Estas caracter\'isticas demostram uma alternativa interessante para estes equipamentos limitados. Testes feitos em implementa\c{c}\~oes de sistemas operacionais similares ao EPOS, como Contiki e TinyOS, utilizando o protocolo CoAP demonstram redu\c{c}\~ao no consumo de energia e mem\'oria em rela\c{c}\~ao ao HTTP.\cite{} .

A falta de padroniza\c{c}\~ao dos protocolos afeta o desenvolvimento de uma rede p\'ublica ub\'iqua de uma cidade inteligente por exemplo. Grande parte das solu\c{c}\~oes utiliza protocolos propriet\'arios, que se comunicacam apenas com os produtos de um mesmo fabricante.

O protocolo HTTP foi desenvolvido para comunica\c{c}\~ao de computadores de prop\'osito geral, onde as restri\c{c}\~oes citadas n\~ao s\~ao comuns. Em rela\c{c}\~ao ao tamanho, o pacote HTTP \'e um problema para redes ZigBEE, j\'a que estas redes que possuem uma restri\c{c}\~ao de 128 bytes. O protocolo TCP precisa transmitir mensagens adicionais para manter uma conex\~ao, outra caracter\'istica que n\~ao \'e interessante para RSSF.

Um protocolo leve como CoAP pode tornar vi\'avel a cria\c{c}\~ao de aplica\-\c{c}\~oes web em redes de sensores sem fio por um baixo custo. Neste trabalho \'e proposto uma infraestrutura de comunica\c{c}\~ao entre redes de sensores sem fio e a Internet, utilizando protocolos leves entre os n\'os sensores e um gateway GPRS para \'areas sem acesso \`a WIFI, aproveitando a vasta abrang\^encia da tecnologia de telefonia. Com a utiliza\c{c}\~ao do CoAP \'e esperado uma redu\c{c}\~ao de consumo de energia e mem\'oria, em rela\c{c}\~ao a outros protocolos de aplica\c{c}\~ao existentes.

Em lugares aonde n\~ao existe o acesso a rede cabeada ou sem fio, como lugares afastados, na \'area rural, por exemplo a distribui\c{c}\~ao da informa\c{c}\~ao para Internet ser\'a feita atrav\'es de um gateway.

O gateway ser\'a composto por um EposMoteII e um m\'odulo GPRS ZigBEE/GPRS, respons\'avel por fazer a ponte entre a rede de sensores e a Internet. Atualmente o padr\~ao GPRS oferece a maior cobertura dentre as tecnologias de transmiss\~ao de telefonia no Brasil, atingindo cerca de 5477 munic\'ipios.\cite{CoberturaGPRS}


\section{Metodologia}

Ser\'a feito um levantamento dos componentes necess\'arios para o desenvolvimento do gateway 6LowPan/GPRS utilizando o mote EPOSMote II e um modem GPRS, que ser\'a definido no decorrer do trabalho. Ent\~ao ser\'a desenvolvido o esquem\'atico para fabrica\c{c}\~ao do gateway.

Ap\'os a valida\c{c}\~ao do modelo, ser\'a iniciada a prototipa\c{c}\~ao do hardware e a implementa\c{c}\~ao do protocolo de aplica\c{c}\~ao CoAP no EPOS. O desenvolvimento do protocolo ser\'a orientado a testes, aonde o c\'odigo escrito apenas satisfaz as condi\c{c}\~oes necess\'arias para validar um comportamento desejado da aplica\c{c}\~ao.

Nos testes de integra\c{c}\~ao do gateway, ser\'a utilizada uma placa de desenvolvimento em conjunto com um m\'odulo M95 da Quectel disponibilizada pelo Laborat\'orio. Os testes de envio de mensagens em diversos protocolos, inclusive testes com comandos propriet\'arios adicionais do modem.

Para testes de integra\c{c}\~ao, as aplica\c{c}\~oes ser\~ao executados na plataforma de sensores sem fio EPOS Mote II utilizando o EPOS com o CoAP desenvolvido.

O m\'odulo de integra\c{c}\~ao composto pelo mote, modem GPRS e o circuito necess\'ario para o funcionamento ser\~ao desenvolvidas em paralelo por um colega do laborat\'orio.
