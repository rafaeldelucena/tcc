Este trabalho implementa uma biblioteca que utiliza a camada UDP do EPOS para dar suporte ao protocolo CoAP, uma aplica\c{c}\~ao gateway GPRS / 802.14.5 utilizando o EPOS e um componente de hardware GPRS que ser\'a acoplado ao EposMoteII.

Durante o desenvolvimento foram realizados diversos estudos para escolher m\'odulo GPRS adequado \`a tarefa e o trabalho necess\'ario para acoplar o protoloco. Testes de valida\c{c}\~ao dos sistemas de software e valida\c{c}\~ao do m\'odulo GPRS foram realizados. Foi realizado um levantamento de requisitos para porte de uma biblioteca CoAP para o EPOS (libCoap, libCantCoap, microCoap, entre outras) e EPOSMoteII. Utilizando testes para validar o funcionamento entre diferentes arquiteturas e compiladores. A execu\c{c}\~ao dos testes foi feita no Qemu.

Foram implementados os mecanismos de transmiss\~ao para mensagens confirm\'aveis e n\~ao-confim\'aveis, requisi\c{c}\~ao e resposta, e suporte a inser\c{c}\~ao de recursos, como sensores e atuadores como servi\c{c}os CoAP.

A aplica\c{c}\~ao respons\'avel pelo roteamento de mensagens para Internet utiliza a tecnologia GPRS, provida por um m\'odulo GSM/GPRS da Quectel o M95.

\section{Levantamento de Requisitos}
\'E um requisito geral deste trabalho definir uma infraestrutura flex\'ivel para a constru\c{c}\~ao de aplica\c{c}\~oes embarcadas em modelo de webservices utilizando redes de sensores sem fio.

O usu\'ario ir\'a acessar a rede de sensores sem fio por uma aplica\c{c}\~ao html5 hospedada na Internet aonde \'e poss\'ivel enviar requisi\c{c}\~oes para a rede de sensores de teste e listar os servi\c{c}os oferecidos e exibir as repostas.

\subsection{Requisitos Funcionais}
S\~ao requisitos funcionais da solu\c{c}\~ao coletar informa\c{c}\~ao do ambiente atrav\'es de sensores e transmit\'i-las atrav\'es da Internet e f\'acil integra\c{c}\~ao com a Internet mesmo em locais sem rede WIFI.

As principais fun\c{c}\~oes deste gateway s\~ao receber os dados da rede de sensores e encaminh\'a-las para um servidor remoto que armazenar\'a essas informa\c{c}\~oes e exibir\'a de forma conveniente para o usu\'ario final.

Ser\'a poss\'ivel comunicar-se em tempo real com a rede de sensores, utilizando um m\'odulo GPRS que ir\'a repassar as requisi\c{c}\~oes e respostas alimentadas pelo usu\'ario.

As fun\c{c}\~oes da da aplica\c{c}\~ao do gateway s\~ao:
\begin{enumerate}
    \item Configura\c{c}\~ao, envio e recebimento de SMS;
    \item Configura\c{c}\~ao de contexto PDP, Configura\c{c}\~ao GPRS;
    \item Configura\c{c}\~ao UDP/IP no enlace GPRS
    \item Configura\c{c}\~ao TCP/IP e manuten\c{c}\~ao de conex\~ao TCP/IP no enlace GPRS
    \item Recebimento de requisi\c{c}\~oes CoAP.
\end{enumerate}

\subsection{Requisitos N\~ao Funcionais}

Os webservices que v\~ao executar nos motes devem herdar a caracteristica de baixo consumo energ\'etico para que possam durar por anos, e serem extens\'iveis, podendo ser reutilizada em outras arquiteturas.

Al\'em disso os servi\c{c}os ser\~ao listados utilizando o padr\~ao \cite{rfc6690} disponibilizados na forma de webservices CoAP. A padroniza\c{c}\~ao na comunica\c{c}\~ao visa facilitar a interconex\~ao dos sistemas de diversas plataformas.

Caracter\'isticas herdadas dos n\'os simplificados s\~ao:
\begin{description}
\item[Armazenamento:] deve ser suficientemente pequeno para ser utilizado em microcontroladores.
\item[Energia:] consumir pouca energia para longa durabilidade.
\item[Valor:] utilizar uma infraestrutura de hardware simples para realizar as tarefas.
\end{description}

\section{Especifica\c{c}\~ao}
\subsection{Arquitetura}

A aplica\c{c}\~ao \'e composta pelos n\'os webservers CoAP, um n\'o cliente CoAP que far\'a o roteamento para Internet utilizando um m\'odulo GPRS.  Os webservers informam a temperatura, atrav\'es de respostas a requisi\c{c}\~oes CoAP. A figura \ref{arquitetura} ilustra a interconex\~ao entre os nodos da rede.

\begin{figure}[H]
   \label{arquitetura}
   \centering
   \includegraphics[width=0.8\textwidth]{figuras/arquitetura.png}
   \caption{Vis\~ao geral sobre comunica\c{c}\~ao do sistema.}
\end{figure}

\subsection{Componentes}
A aplica\c{c}\~ao do gateway \'e composta por: Mecanismos de temporiza\c{c}\~ao, camada UDP/IP, parser de pacote CoAP, conjunto de comandos AT, Estruturas de filas, Hash simples e Threads.

A implementa\c{c}\~ao consiste num m\'odulo que trata requisi\c{c}\~oes, encapsula em pacotes e transmite por mecanimos de transmiss\~ao baseados em \cite{draft-ietf-core-coap-18}.

A biblioteca utilizada para montar o pacote CoAP foi a biblioteca CantCoap, na foram realizados algumas corre\c{c}\~oes e testes para facilitar a verifica\c{c}\~ao da execu\c{c}\~ao correta dos algoritmos internos durante o desenvolvimento da aplica\c{c}\~ao. As altera\c{c}\~oes resultaram em contribui\c{c}\~ao para o referido projeto, aceita pelo mantenedor.

Para o funcionamento desta biblioteca no EPOS, e para utilizar uma MTU limitada a 128 bytes \'e utilizado um buffer com um valor m\'aximo e armazenado os dados do pacote no buffer. Foi necess\'ario alterar os tipos das vari\'aveis para se adquerem ao EPOS.

Foi desenvolvimento de um mecanismo de retransmiss\~ao de mensagens n\~ao-confirmadas utilizando uma lista ordenada. No mecanismo de requisic\c{c}\~o e resposta, as requisi\c{c}\~oes pendentes foram armezenadas num Hash com a chave sendo o token gerado pelo cliente.

O protocolo CoAP foi modelado conforme \'e mostrado na figura \ref{uml} abaixo:
\begin{figure}[H]
   \label{uml}
   \centering
   \includegraphics[width=0.7\textwidth]{figuras/uml.png}
   \caption{Diagrama UML das entidades de software implementadas.}
\end{figure}

A figura \ref{casodeuso} mostra o diagrama de caso de uso das principais fun\c{c}\~oes desenvolvidas.
\begin{figure}[H]
   \label{casodeuso}
   \centering
   \includegraphics[width=0.7\textwidth]{figuras/casodeuso.png}
   \caption{Diagrama de casos de uso.}
\end{figure}

A aplica\c{c}\~ao cliente foi desenvolvida em um Linux utilizando tecnologias HTML5, JavaScript, JSON e HTML foram utilizados. O servidor HTTP utilizado foi HttpServer do NodeJS, que recebe requisi\c{c}\~oes HTTP dos clientes para a rede de sensores sem fio utilizando a biblioteca Node-CoAP.

A aplica\c{c}\~ao desenvolvida no EPOS utiliza buffer para o recebimento de dados da rede 802.15.4 que ser\'a enviado para rede via GPRS. Foram utilizadas duas threads, uma produtora que ficar\'a escutando o r\'adio 802.15.4 e outra consumidora que ser\'a respons\'avel em utlizar estes dados na rede de sensores e encaminh\'a-los pra Internet usando a extens\~ao GPRS do EPOSmote II. Para validar o comportamento foram utilizados testes da pr\'opria biblioteca CoAP portados para o EPOS.

\section{Testes}

Durante o desenvolvimento foram realizados in\'umeros testes para verificar e validar o correto comportamento dos componentes de software e hardware.  A implementa\c{c}\~ao do protocolo CoAP foi testada da seguinte maneira:

\begin{description}
    \item[Testes unit\'arios:] Testes de constru\c{c}\~ao de pacotes v\'alidos e inv\'alidos utilizando como entrada sequ\^encia de caracteres e tratamento de respostas v\'alidas e inv\'lidas.
    \item[Testes de integra\c{c}\~ao:] Testes de interoperabilidade entre as implementa\c{c}\~oes, utilizando cen\'arios parecidos com o IOT Plugtest.
\end{description}

Os procedimentos para testar o m\'odulo GPRS foram: Enviar e recebimento de mensagens; criar socket UDP e TCP, enviar e receber mensagem via socket, fazer requisi\c{c}\~ao e receber respostas HTTP.
