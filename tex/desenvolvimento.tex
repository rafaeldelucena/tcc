Este trabalho implementa uma biblioteca que utiliza a camada UDP do EPOS para dar suporte ao protocolo CoAP, uma aplica\c{c}\~ao gateway GPRS/802.14.5 utilizando o EPOS e um componente de hardware GPRS que ser\'a acoplado ao EposMoteII.

Para isto foram realizados diversos estudos durante o desenvolvimento do protocolo, pesquisas para escolher um m\'odulo GPRS adequado a tarefa. Al\'em disso testes de valida\c{c}\~ao dos sistemas de software e valida\c{c}\~ao do m\'odulo GPRS, levantamento de requisitos para porte de uma biblioteca CoAP para o EPOS (libCoap, libCantCoap, microCoap, entre outras) e EPOSMoteII.

Porte da biblioteca para o EPOS. Utilizando testes para validar o funcionamento entre diferentes arquiteturas e compiladores. A execu\c{c}\~ao dos testes foi feita no Qemu.

Implementa\c{c}\~ao dos mecanismos de transmiss\~ao para mensagens confirm\'aveis e n\~ao-confim\'aveis, requisi\c{c}\~ao e resposta.

A aplica\c{c}\~ao que ser\'a respons\'avel pelo roteamento de mensagens para Internet utiliza a tecnologia GPRS, provida por um m\'odulo GSM/GPRS da Quectel o M95.

\section{Levantamento de Requisitos}
Infraestrutura flex\'ivel para a contru\c{c}\~ao de aplica\c{c}\~oes embarcadas em modelo de webservices utilizando redes de sensores sem fio.

O usu\'ario ir\'a acessar a rede de sensores sem fio por uma aplica\c{c}\~ao html5, hospedada em http://minhascoisas.io.
Nesta aplica\c{c}\~ao \'e poss\'ivel enviar requisi\c{c}\~oes para a rede de sensores de teste e listar os servi\c{c}os oferecidos.

\subsection{Requisitos Funcionais}
Coletar informa\c{c}\~ao do ambiente atrav\'es de sensores e transmit\'i-las atrav\'es da Internet. F\'acil integra\c{c}\~ao com a Internet mesmo em locais sem rede WIFI.

As principais fun\c{c}\~oes deste gateway s\~ao receber os dados da rede de sensores e encaminh\'a-las para um servidor remoto que armazenar\'a essas informa\c{c}\~oes e exibir\'a de forma conveniente para o usu\'ario final.

Ser\'a poss\'ivel comunicar-se em tempo real com a rede de sensores, utilizando um m\'odulo GPRS que ir\'a repassar as requisi\c{c}\~oes e respostas alimentadas pelo usu\'ario.

As fun\c{c}\~oes a da aplica\c{c}\~ao do gateway s\~ao:
\begin{enumerate}
    \item Configura\c{c}\~ao, envio e recebimento de SMS;
    \item Configura\c{c}\~ao contexto PDP, Configura\c{c}\~ao GPRS;
    \item Configura\c{c}\~ao TCP/IP e manuten\c{c}\~ao de conex\~ao TCP/IP.
\end{enumerate}

\subsection{Requisitos N\~ao Funcionais}

Os webservices que v\~ao executar nos motes devem herdar a caracteristica de baixo consumo energ\'etico para que possam durar por anos, e serem extens\'iveis, podendo ser reutilizada em outras arquiteturas.

Al\'em os servi\c{c}os ser\~ao listatos utilizando o padr\~ao \cite{rfc6690} disso os dados captados por sensores ser\~ao disponibilizados na forma de webservices CoAP, protocolo espec\'ifico para este tipo de aplicac\c{c}\~ao.

A padroniza\c{c}\~ao na comunica\c{c}\~ao visa facilitar a interconex\~ao dos sistemas de diversas plataformas.

Caracter\'isticas destes sistemas s\~ao eficiente em:
    \begin{itemize}
        \item Armazenamento: deve ser suficientemente pequeno para ser utilizado em microcontroladores.
        \item Energia: cosumir pouca energia para longa durabilidade com bateria.
        \item Valor: utilizar uma infraestrutura de hardware simples para realizar as tarefas.
    \end{itemize}


\section{Especifica\c{c}\~ao}
\subsection{Arquitetura}

A pilha utilizada \'e composta por EposMoteII, EPOS, UDP, CoAP, conforme a figura \ref{arquitetura} abaixo. 

\subsection{Componentes}
O CoAP implementado \'e composto pelos seguintes componentes:




A implementa\c{c}\~ao consiste num m\'odulo que trata requisi\c{c}\~oes, encapsula em pacotes e transmite por mecanimos de transmiss\~ao baseados em \cite{draft-ietf-core-coap-18}.

A biblioteca utilizada para montar o pacote CoAP foi:\\https://github.com/staropram/cantcoap.git. Na qual enviei algumas corre\c{c}\~oes e testes para facilitar a verifica\c{c}\~ao da execu\c{c}\~ao correta dos algoritmos internos durante mudan\c{c}as no c\'odigo. As altera\c{c}\~oes podem ser visualisadas aqui:\\https://github.com/staropram/cantcoap/commits?author=rafaeldelucena.

Para o funcionamento desta biblioteca no EPOS, e para utilizar uma MTU limitada a 128 bytes utilizo um buffer com um valor m\'aximo e armazeno os dados do pacote no buffer. Foi necess\'ario alterar os tipos das vari\'aveis para se adquerem ao EPOS.

O desenvolvimento de um mecanismo de retransmiss\~ao de mensagens n\~ao-confirmadas utilizando uma lista ordenada. Mecanismo de requisic\c{c}\~o e resposta, as requisi\c{c}\~oes pendentes foram armezenadas num Hash com a chave sendo o token gerado pelo cliente.

Para validar o comportamento utilizei alguns testes da pr\'opria biblioteca CoAP portados para o EPOS. Foi necess\'ario implementar a fun\c{c}\~ao assert. J\'a que seria bem mais trabalhoso adicionar uma ferramenta de testes no sistema de build do EPOS.

\section{Testes}
Testes do cliente:
Fazendo uma requisi\c{c}\~ao confirm\'aveis e n\~ao-confirm\'aveis do tipo: GET, POST, PUT, DELETE.
Recebendo respostas: válidas e inválidas.


Testes do Servidor:
Recebendo e respondendo requisi\c{c}\~oes: que possui recurso, que n\~ao possui, descoberta de recurso.

\subsection{Testes na placa de desenvolvimento do m\'ouldo GPRS}
Os testes feitos foram: Enviar e recebimento de mensagens; criar socket TCP, enviar e receber mensagem via socket, fazer requisi\c{c}\~ao HTTP, foi poss\'ivel utilizando os comandos propriet\'arios do modem.
