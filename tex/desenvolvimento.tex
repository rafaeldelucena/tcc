Este trabalho prop\~oe a implementa\c{c}\~ao de uma biblioteca que utiliza a camada UDP do EPOS para dar suporte ao protocolo CoAP. O trabalho tamb\'em consiste na implementa\c{c}\~ao de uma aplica\c{c}\~ao para gateway GPRS/Zigbee utilizando o EPOS e um componente de hardware que ser\'a acoplado ao EposMoteII.

Foram realizados diversos estudos durante o desenvolvimento do protocolo e testes para valida\c{c}\~ao do m\'odulo GPRS. e durante o levamento de requisitos para portar a libcoap para o EPOS, e EPOSMoteII. Levantar requisitos libcoap no EPOS, requisitos do HW.

O desenvolvimento da aplica\c{c}\~ao no EPOS que ser\'a respons\'avel pelo roteamento de mensagens para Internet utiliza a tecnologia GPRS, provida por um m\'odulo GSM/GPRS da Quectel o M95.

Porte da biblioteca para o EPOS. Utilizando testes para validar o funcionamento entre diferentes arquiteturas e compiladores. A execu\c{c}\~ao dos testes foi feita no Qemu.

Implementa\c{c}\~ao do CoAP: mecanismos de retransmiss\~ao, requisi\c{c}\~ao e resposta. Envio de mensagens confirm\'aveis e n\~ao confirm\'aveis

\section{An\'alise de Requisitos}
Infraestrutura flex\'ivel para a contru\c{c}\~ao de aplica\c{c}\~oes embarcadas em modelo de webservices utilizando redes de sensores sem fio.
\subsection{Requisitos Funcionais}{
Coletar informa\c{c}\~ao do ambiente atrav\'es de sensores e transmit\'i-las atrav\'es da Internet. F\'acil integra\c{c}\~ao com a Internet mesmo em locais sem rede WIFI.

As principais fun\c{c}\~oes deste gateway s\~ao receber os dados da rede de sensores e encaminh\'a-las para um servidor remoto que armazenar\'a essas informa\c{c}\~oes e exibir\'a de forma conveniente para o usu\'ario final.
As fun\c{c}\~oes a da aplica\c{c}\~ao do gateway s\~ao:
\begin{enumerate}
    \item Configura\c{c}\~ao, envia e recebimentode SMS;
    \item Configura\c{c}\~ao contexto PDP, Configura\c{c}\~ao GPRS;
    \item Configura\c{c}\~ao TCP/IP e manuten\c{c}\~ao de conex\~ao TCP/IP.
\end{enumerate}
}

\subsection{Requisitos N\~ao Funcionais}{

Extens\'ivel: Podendo ser reutilizada em diversas arquiteturas.
Padroniza\c{c}\~ao na comunica\c{c}\~ao para facilitar a interconex\~ao dos sistemas.
Eficiente em:
    \begin{itemize}
        \item Energia: cosumir pouca energia para longa durabilidade com bateria.
        \item Valor: utilizar uma infraestrutura de hardware simples para realizar as tarefas.
    \end{itemize}
}

\section{Implementaca\c{c}\~ao}
\section{Testes}
\subsection{Testes na placa de desenvolvimento do m\'ouldo GPRS}{
    Os testes feitos foram: Enviar e recebimento de mensagens; criar socket TCP, enviar e receber mensagem via socket, fazer requisi\c{c}\~ao HTTP, foi poss\'ivel utilizando os comandos propriet\'arios do modem.}
