%%%%%%%%%%%%%%%%%%%%%%%%%%%%%%%%%%%%%%%%%%%%%%%%%%%%%%%%%%%%%%%%%%%%%%%
% Universidade Federal de Santa Catarina             
% Biblioteca Universitária                     
%----------------------------------------------------------------------
% Exemplo de utilização da documentclass ufscThesis
%----------------------------------------------------------------------
% (c)2013 Roberto Simoni (roberto.emc@gmail.com)
%         Carlos R Rocha (cticarlo@gmail.com)
%         Rafael M Casali (rafaelmcasali@yahoo.com.br)
%%%%%%%%%%%%%%%%%%%%%%%%%%%%%%%%%%%%%%%%%%%%%%%%%%%%%%%%%%%%%%%%%%%%%%%
\documentclass{ufscThesis} % Definicao do documentclass ufscThesis	

%----------------------------------------------------------------------
% Pacotes usados especificamente neste documento
\usepackage{graphicx} % Possibilita o uso de figuras e gráficos
\usepackage{color}    % Possibilita o uso de cores no documento
\usepackage{listings}
\usepackage{amsmath}
%----------------------------------------------------------------------
% Comandos criados pelo usuário
\newcommand{\afazer}[1]{{\color{red}{#1}}} % Para destacar uma parte a ser trabalhada

%----------------------------------------------------------------------
% Identificadores do trabalho
% Usados para preencher os elementos pré-textuais
\instituicao[a]{Universidade Federal de Santa Catarina} % Opcional
\departamento[a]{Departamento de Informática e Estatística}
\curso[o]{Bacharel em Ciências da Computação}
\documento[a]{Trabalho de conclusão de curso} % [o] para dissertação [a] para tese
\titulo{Implementação do protocolo CoAP para o monitoramento em redes de sensores sem fio}
\autor{Rafael de Lucena Valle}
\local{Florianópolis} % Opcional (Florianópolis é o padrão)
\data{15}{julho}{2013}
\orientador[Orientador]{Prof. Dr. Antônio Augusto Fröhlich}
\coorientador[Coorientador]{Prof. M.Sc. Arliones Hoeller Jr}
\coordenador[Coordenador]{Prof. Dr. Roberto Cislaghi}

\numerodemembrosnabanca{4} % Isso decide se haverá uma folha adicional
\orientadornabanca{sim} % Se faz parte da banca definir como sim
\coorientadornabanca{sim} % Se faz parte da banca definir como sim
\bancaMembroA{Prof. M.Sc. Arliones Hoeller Jr} %Nome do presidente da banca
\bancaMembroB{Prof. Dr. Antônio Augusto Fröhlich} % Nome do membro da Banca
\bancaMembroC{Prof. Dr. Eng. Rafael Luiz Cancian} % Nome do membro da Banca
\bancaMembroD{Prof. Dr. Frank Siqueira} % Nome do membro da Banca

\agradecimento{Inserir os agradecimentos aos colaboradores à execução do trabalho.}

\textoResumo {Redes de sensores são utilizadas para a captação, processamento de informação e atuação sobre um ambiente, tornando-as importantes para controle, telemetria e rastreamento de sistemas. Os nós das redes geralmente são computadores e rádios simplificados, que possuem restrições de memória, processamento, energia e comunicação, mas um custo relativamente baixo de equipamentos, tornando interessante a implantação destes sistemas. O protocolo HTTP foi desenvolvido pensado em computadores de propósito geral, onde essas restrições não existem. Um protocolo leve como CoAP pode tornar viável a criação de aplicações web em redes de sensores sem fio por um baixo custo.
É proposto uma infraestrutura de comunicação entre redes de sensores sem fio e a Internet, utilizando protocolos leves entre os nós sensores e um gateway que utiliza GPRS para áreas sem acesso à WIFI, aproveitando a vasta abrangência da tecnologia de telefonia. Com a Utilização do CoAP é esperado uma redução de consumo de energia e memória, em relação a outros protocolos de aplicação existentes.}

\palavrasChave { internetworking wireless sensor networks IPv6 6LoWPAN GPRS CoAP }

%----------------------------------------------------------------------
% Início do documento                                
\begin{document}
%--------------------------------------------------------
% Elementos pré-textuais
\capa}
\folhaderosto%[comficha] % Se nao quiser imprimir a ficha, é só não usar o parâmetro
\folhaaprovacao
%\paginadedicatoria
%\paginaagradecimento
%\paginaepigrafe
\paginaresumo
%\paginaabstract
%\pretextuais % Substitui todos os elementos pre-textuais acima
\listadefiguras % as listas dependem da necessidade do usuário
%\listadetabelas 
\listadeabreviaturas
%\listadesimbolos
\sumario
%--------------------------------------------------------
% Elementos textuais

\chapter{Introdução}
Redes de sensores s\~ao utilizadas para a capta\c{c}\~ao, processamento de informa\c{c}\~ao e atua\c{c}\~ao sobre um ambiente, tornando-as importantes em aplica\c{c}\~oes de controle, telemetria e rastreamento de sistemas.

N\'os que participam destas redes geralmente s\~ao computadores e r\'adios simplificados, que possuem restri\c{c}\~oes de mem\'oria, processamento, energia e capacidade de comunica\c{c}\~ao, mas um custo relativamente baixo de equipamentos.

O maior consumo de energia neste tipo de aplica\c{c}\~ao \'e do r\'adio, portanto os desafios dos algoritmos de comunica\c{c}\~ao nesta \'area s\~ao manter os r\'adios ligados o m\'inimo de tempo poss\'ivel sem comprometer a conectividade do n\'o.

\section{Objetivos}
O objetivo geral desse trabalho \'e descrever e implementar webservices em uma rede sensores sem fio, que far\~ao\
a aquisi\c{c}\~ao dos dados do ambiente e disponibilizar\~ao as informa\c{c}\~oes captadas na Internet.

\subsection{Objetivos Espec\'ificos}
Os objetivos espec\'ificos do trabalho s\~ao relacionados ao desenvolvimento de duas aplica\c{c}\~oes e uma biblioteca.

%TODO Melhorar!

\begin{itemize}
    \item Portar o protocolo CoAP, Constrained Application Protocol para o EPOS, Embedded Paralell Operating System.
    \item Implementar uma aplica\c{c}\~ao de redes de sensores sem fio.
    \item Desenvolver o firmware do  gateway GPRS/Zigbee, que ir\'a disponibilizar as informa\c{c}\~oes captadas da rede para Internet.
\end{itemize}
        

\section{Justificativa}

Os mecanismos de confiabilidade na transmiss\~ao de dados, t\'ecnicas para se manter uma conex\~ao do TCP e rearranjos que s\~ao feitos para garantir a ordem das mensagens recebidas n\~ao s\~ao adequados para um dispositivo que dependam de uma bateria, pois podem fazer que fiquem com seus transmissores, ligados por mais tempo para manter a conex\~ao ou at\'e mesmo para reenvio de mensagens. O maior consumo de energia de um n\'o sensor \'e no envio e recebimento de dados, quando mantem seu transmissor ligado.

Assim faz-se uso do UDP, um protocolo que n\~ao mant\'em conex\~ao onde os dados s\~ao recebidos fora de ordem e o envio \'e feito de uma mensagem por vez, sem o uso de streammings do TCP, que fazem que quem receba a mensagem precise mont\'a-la e garantir que nenhuma das pe\c{c}as est\'a corrompida. Isto implica tamb\'em na redu\c{c}\~ao do tamanho do cabe\c{c}alho do pacote. Estas caracter\'isticas demostram uma alternativa interessante para estes equipamentos limitados. Testes feitos em implementa\c{c}\~oes de sistemas operacionais similares ao EPOS, como Contiki e TinyOS, utilizando o protocolo CoAP demonstram redu\c{c}\~ao no consumo de energia e mem\'oria em rela\c{c}\~ao ao HTTP.\cite{}

A falta de padroniza\c{c}\~ao dos protocolos afeta o desenvolvimento de uma rede p\'ublica ub\'iqua de uma cidade inteligente por exemplo. Grande parte das solu\c{c}\~oes utiliza protocolos propriet\'arios, que se comunicacam apenas com os produtos de um mesmo fabricante.

O protocolo HTTP foi desenvolvido para comunica\c{c}\~ao de computadores de prop\'osito geral, onde as restri\c{c}\~oes citadas n\~ao s\~ao comuns. Em rela\c{c}\~ao ao tamanho, o pacote HTTP \'e um problema para redes ZigBEE, j\'a que estas redes que possuem uma restri\c{c}\~ao de 128 bytes. O protocolo TCP precisa transmitir mensagens adicionais para manter uma conex\~ao, outra caracter\'istica que n\~ao \'e interessante para RSSF.

Um protocolo leve como CoAP pode tornar vi\'avel a cria\c{c}\~ao de aplica\-\c{c}\~oes web em redes de sensores sem fio por um baixo custo. Neste trabalho \'e proposto uma infraestrutura de comunica\c{c}\~ao entre redes de sensores sem fio e a Internet, utilizando protocolos leves entre os n\'os sensores e um gateway GPRS para \'areas sem acesso \`a WIFI, aproveitando a vasta abrang\^encia da tecnologia de telefonia. Com a utiliza\c{c}\~ao do CoAP \'e esperado uma redu\c{c}\~ao de consumo de energia e mem\'oria, em rela\c{c}\~ao a outros protocolos de aplica\c{c}\~ao existentes.

Em lugares aonde n\~ao existe o acesso a rede cabeada ou sem fio, como lugares afastados, na \'area rural, por exemplo a distribui\c{c}\~ao da informa\c{c}\~ao para Internet ser\'a feita atrav\'es de um gateway. Este gateway ser\'a composto por um EposMoteII e um m\'odulo GPRS.
Atualmente o padr\~ao GPRS oferece a maior cobertura dentre as tecnologias de transmiss\~ao de telefonia no Brasil, atingindo cerca de 5477 munic\'ipios.

\section{Metodologia}

Ser\'a feito um levantamento dos componentes necess\'arios para o desenvolvimento do gateway 6LowPan/GPRS utilizando o mote EPOSMote II e um modem GPRS, que ser\'a definido no decorrer do trabalho. Ent\~ao ser\'a desenvolvido o esquem\'atico para fabrica\c{c}\~ao do gateway.

Ap\'os a valida\c{c}\~ao do modelo, ser\'a iniciada a prototipa\c{c}\~ao do hardware e a implementa\c{c}\~ao do protocolo de aplica\c{c}\~ao CoAP no EPOS. O desenvolvimento do protocolo ser\'a orientado a testes, aonde o c\'odigo escrito apenas satisfaz as condi\c{c}\~oes necess\'arias para validar um comportamento desejado da aplica\c{c}\~ao.

Nos testes de integra\c{c}\~ao do gateway, ser\'a utilizada uma placa de desenvolvimento em conjunto com um m\'odulo M95 da Quectel disponibilizada pelo Laborat\'orio, para serem feitos os testes de envio de mensagens em diversos protocolos, inclusive testes com comandos propriet\'arios adicionais do modem.

Para testes de integra\c{c}\~ao, as aplica\c{c}\~oes ser\~ao executados na plataforma de sensores sem fio EPOS Mote II utilizando o EPOS com o CoAP desenvolvido.


\chapter{Desenvolvimento}
Este trabalho implementa uma biblioteca que utiliza a camada UDP do EPOS para dar suporte ao protocolo CoAP, uma aplica\c{c}\~ao gateway GPRS / 802.14.5 utilizando o EPOS e um componente de hardware GPRS que ser\'a acoplado ao EposMoteII.

Durante o desenvolvimento foram realizados diversos estudos para escolher m\'odulo GPRS adequado a tarefa e o trabalho necess\'ario para acoplar o protoloco. Testes de valida\c{c}\~ao dos sistemas de software e valida\c{c}\~ao do m\'odulo GPRS foram realizados. Foi realizado um levantamento de requisitos para porte de uma biblioteca CoAP para o EPOS (libCoap, libCantCoap, microCoap, entre outras) e EPOSMoteII. Utilizando testes para validar o funcionamento entre diferentes arquiteturas e compiladores. A execu\c{c}\~ao dos testes foi feita no Qemu.

Implementa\c{c}\~ao dos mecanismos de transmiss\~ao para mensagens confirm\'aveis e n\~ao-confim\'aveis, requisi\c{c}\~ao e resposta, e suporte a inser\c{c}\~ao de recursos, como sensores e atuadores como servi\c{c}os CoAP.

A aplica\c{c}\~ao respons\'avel pelo roteamento de mensagens para Internet utiliza a tecnologia GPRS, provida por um m\'odulo GSM/GPRS da Quectel o M95.

\section{Levantamento de Requisitos}
Infraestrutura flex\'ivel para a contru\c{c}\~ao de aplica\c{c}\~oes embarcadas em modelo de webservices utilizando redes de sensores sem fio.

O usu\'ario ir\'a acessar a rede de sensores sem fio por uma aplica\c{c}\~ao html5, hospedada em na internet no endere\c{c}o: http://adefinir. Nesta aplica\c{c}\~ao \'e poss\'ivel enviar requisi\c{c}\~oes para a rede de sensores de teste e listar os servi\c{c}os oferecidos.

\subsection{Requisitos Funcionais}
Coletar informa\c{c}\~ao do ambiente atrav\'es de sensores e transmit\'i-las atrav\'es da Internet. F\'acil integra\c{c}\~ao com a Internet mesmo em locais sem rede WIFI.

As principais fun\c{c}\~oes deste gateway s\~ao receber os dados da rede de sensores e encaminh\'a-las para um servidor remoto que armazenar\'a essas informa\c{c}\~oes e exibir\'a de forma conveniente para o usu\'ario final.

Ser\'a poss\'ivel comunicar-se em tempo real com a rede de sensores, utilizando um m\'odulo GPRS que ir\'a repassar as requisi\c{c}\~oes e respostas alimentadas pelo usu\'ario.

As fun\c{c}\~oes a da aplica\c{c}\~ao do gateway s\~ao:
\begin{enumerate}
    \item Configura\c{c}\~ao, envio e recebimento de SMS;
    \item Configura\c{c}\~ao contexto PDP, Configura\c{c}\~ao GPRS;
    \item Configura\c{c}\~ao TCP/IP e manuten\c{c}\~ao de conex\~ao TCP/IP.
    \item Recebimento de requisi\c{c}\~oes CoAP.
\end{enumerate}

\subsection{Requisitos N\~ao Funcionais}

Os webservices que v\~ao executar nos motes devem herdar a caracteristica de baixo consumo energ\'etico para que possam durar por anos, e serem extens\'iveis, podendo ser reutilizada em outras arquiteturas.

Al\'em os servi\c{c}os ser\~ao listatos utilizando o padr\~ao \cite{rfc6690} disso os dados captados por sensores ser\~ao disponibilizados na forma de webservices CoAP, protocolo espec\'ifico para este tipo de aplicac\c{c}\~ao.

A padroniza\c{c}\~ao na comunica\c{c}\~ao visa facilitar a interconex\~ao dos sistemas de diversas plataformas.

Caracter\'isticas destes sistemas s\~ao eficiente em:
    \begin{itemize}
        \item Armazenamento: deve ser suficientemente pequeno para ser utilizado em microcontroladores.
        \item Energia: cosumir pouca energia para longa durabilidade com bateria.
        \item Valor: utilizar uma infraestrutura de hardware simples para realizar as tarefas.
    \end{itemize}


\section{Especifica\c{c}\~ao}
\subsection{Arquitetura}

A aplica\c{c}\~ao \'e composta pelos n\'os webservers CoAP, um n\'o cliente CoAP que far\'a o roteamento para Internet utilizando um m\'odulo GPRS.  Os webservers informam a temperatura, atrav\'es de respostas a requisic\c{c}\~oes CoAP. A figura \ref{arquitetura} ilustra a interconex\~ao entre os nodos da rede.

\begin{figure}[h]
   \label{arquitetura}
   \centering
   \includegraphics[width=0.8\textwidth]{figuras/arquitetura.png}
   \caption{Vis\~ao geral sobre comunica\c{c}\~ao do sistema.}
\end{figure}

\subsection{Componentes}
A aplica\c{c}\~ao do gateway \'e composta por: Mecanismos de temporiza\c{c}\~ao, camada UDP/IP, parser de pacote CoAP, conjunto de comandos AT, Estruturas de filas, Hash simples e Threads.

A implementa\c{c}\~ao consiste num m\'odulo que trata requisi\c{c}\~oes, encapsula em pacotes e transmite por mecanimos de transmiss\~ao baseados em \cite{draft-ietf-core-coap-18}.

A biblioteca utilizada para montar o pacote CoAP foi:\\https://github.com/staropram/cantcoap.git. Na qual enviei algumas corre\c{c}\~oes e testes para facilitar a verifica\c{c}\~ao da execu\c{c}\~ao correta dos algoritmos internos durante mudan\c{c}as no c\'odigo. As altera\c{c}\~oes podem ser visualisadas aqui:\\https://github.com/staropram/cantcoap/commits?author=rafaeldelucena.

Para o funcionamento desta biblioteca no EPOS, e para utilizar uma MTU limitada a 128 bytes utilizo um buffer com um valor m\'aximo e armazeno os dados do pacote no buffer. Foi necess\'ario alterar os tipos das vari\'aveis para se adquerem ao EPOS.

O desenvolvimento de um mecanismo de retransmiss\~ao de mensagens n\~ao-confirmadas utilizando uma lista ordenada. Mecanismo de requisic\c{c}\~o e resposta, as requisi\c{c}\~oes pendentes foram armezenadas num Hash com a chave sendo o token gerado pelo cliente.

O protocolo CoAP foi modelado conforme \'e mostrado na figura \ref{uml} abaixo:
\begin{figure}[h]
   \label{uml}
   \centering
   \includegraphics[width=0.9\textwidth]{figuras/uml.png}
   \caption{Diagrama UML das entidades de software implementadas.}
\end{figure}

A figura \ref{casodeuso} mostra o diagrama de caso de uso das principais fun\c{c}\~oes desenvolvidas.
\begin{figure}[h]
   \label{casodeuso}
   \centering
   \includegraphics[width=0.8\textwidth]{figuras/casodeuso.png}
   \caption{Diagrama de casos de uso.}
\end{figure}

A aplica\c{c}\~ao cliente foi desenvolvida utilizando tecnologias HTML5. JavaScript, Json e HTML foram utilizados.
A biblioteca CoAP foi:\\https://github.com/mcollina/node-coap

A aplica\c{c}\~ao desenvolvida no EPOS utiliza buffer para o recebimento de dados da rede 802.15.4 que ser\'a enviado para rede via GPRS. Duas threads, uma produtora que ficar\'a escutando o r\'adio 802.15.4 e outra consumidora que ser\'a respons\'avel em utlizar estes dados na rede de sensores e encaminh\'a-los pra Internet usando a extens\~ao GPRS do EPOSmote II.

Para validar o comportamento utilizei alguns testes da pr\'opria biblioteca CoAP portados para o EPOS. Foi necess\'ario implementar a fun\c{c}\~ao assert. J\'a que seria bem mais trabalhoso adicionar uma ferramenta de testes no sistema de build do EPOS.

\section{Testes}

Foram realizados in\'umeros testes durante o desenvolvimento para verificar e validar o correto comportamento dos componentes de software e hardware.

Para validar a implementa\c{c}\~ao do protocolo CoAP foram efetuados os seguintes testes:

Testes de constru\c{c}\~ao de pacotes v\'alidos e inv\'alidos utilizando como entrada sequ\^encia de caracteres.

Testes de interoperabilidade entre as implementa\c{c}\~oes, utilizando cen\'arios parecidos com o IOT Plugtest.

Fazendo uma requisi\c{c}\~ao confirm\'aveis e n\~ao-confirm\'aveis do tipo: GET, POST, PUT, DELETE.

Recebendo respostas: v\'alidas e inv\'lidas.

Testes do Servidor:
Recebendo e respondendo requisi\c{c}\~oes: que possui recurso, que n\~ao possui, descoberta de recurso.

Os testes feitos foram: Enviar e recebimento de mensagens; criar socket TCP, enviar e receber mensagem via socket, fazer requisi\c{c}\~ao HTTP, foi poss\'ivel utilizando os comandos propriet\'arios do modem.


\chapter{Considerações Parciais}

\bibliographystyle{ufscThesis/ufsc-alf}
\bibliography{bibliografia}

%--------------------------------------------------------
% Elementos pós-textuais

\anexo
\chapter{Código Desenvolvido}
\lstdefinestyle{customc}{
  belowcaptionskip=1\baselineskip,
  breaklines=true,
  xleftmargin=\parindent,
  language=C++,
  showstringspaces=false,
  basicstyle=\scriptsize\ttfamily,
  keywordstyle=\bfseries\color{green!40!black},
  commentstyle=\itshape\color{blue},
  identifierstyle=\color{blue!20!black},
  stringstyle=\color{red},
}

\subsection{CoAP}

\begin{lstlisting}

#ifndef __coap_packet_h__
#define __coap_packet_h__

#include <coap_pdu.h>

__BEGIN_SYS

class CoapPacket : public CoapPDU
{
    public:
        static const int ackTimeout = 2000000;
        static const double ackRandomFactor = 1.5;
        static const int maxRetransmit = 4;
        static const int nStart = 1;

        CoapPacket(const UDP_Address & to);
        CoapPacket(const UDP_Address & from, const char * data, unsigned len);
        CoapPacket(CoapPDU * pdu);
        CoapPacket();
        ~CoapPacket();

        bool isConfirmable();
        bool isConfirmed();
        bool isFailure() {return false;};
        void setConfirmed();
        void reset();
        virtual void update() {kout << "Packet update" << endl;};
        UDP_Address remote() const;
        int getToken();
    protected:
        void generateNewToken(unsigned int len);
    private:
        static const int _maxPDUSize = 100;
        bool _isConfirmed;
        UDP_Address _remote;
        u8 _pduBuffer[_maxPDUSize];
        Alarm * _alarm;
};

class CoapACK : public CoapPacket
{
    public:
        CoapACK(int id) : CoapPacket() {
            setType(CoapPacket::COAP_ACKNOWLEDGEMENT);
            setCode(CoapPacket::COAP_EMPTY);
            setMessageID(id);
        }
};

class CoapConfirmable : public CoapPacket {
    public:
        CoapConfirmable(CoapPacket::Code code);
        ~CoapConfirmable();
        void update();
        bool isFailure();
        void reset();
    private:
        Alarm * _alarm;
        int _retransmissionCounter;
        int _timeout;
};

__END_SYS

#endif /*__coap_packet_h__*/


#ifndef __coap_request_h__
#define __coap_request_h__

#include <coap_packet.h>

__BEGIN_SYS

class CoapRequest : public CoapConfirmable
{
    public:
        CoapRequest(CoapPacket::Code code, const char* uri);
        virtual void onSuccess();
        virtual void onError();
        static void incomingResponse(CoapPacket * packet);
        bool wasAnswered() { return (_response != 0);}
    private:
        void addResponse(CoapPacket *);
        void addAsPending();
        static CoapRequest * removePendingByToken(int token);
        static unsigned int indexByToken(int token);
        static const int _maxPDUSize = 100;
        static const int _maxPending = 100;
        char _uriBuffer[_maxPDUSize];
        int _uriSize;
        static CoapRequest * _pendingRequests[_maxPending];
        CoapPacket * _response;
};

__END_SYS

#endif /*__coap_request_h__*/


#ifndef __coap_response_h__
#define __coap_response_h__

__BEGIN_SYS

class CoapPacket;

template <class T>
class CoapResponse : public CoapPacket
{
    public:
        CoapResponse(T data, CoapRequest * r);
        ~CoapResponse();
    private:
        T * _data;
};

__END_SYS

#endif /*__coap_response_h__*/


#ifndef __coap_socket_h_
#define __coap_socket_h_

#include <udp.h>
#include <coap_packet.h>
#include <utility/queue.h>

#define COAP_PORT 5683

__BEGIN_SYS

class CoapSocket : public UDP::Socket
{
public:
    CoapSocket(const UDP_Address & local);
    ~CoapSocket();
    void sendPacket(CoapPacket * pdu);
    void sendPacket(const CoapPacket & pdu);
    void received(const UDP_Address & from, const char* msg, unsigned int len);
};

__END_SYS

#endif /* __coap_socket_h_ */


#include <coap_packet.h> 
#include <coap_socket.h>
#include <utility/handler.h>
__BEGIN_SYS

class CoapTransport
{
    public:
    static Function_Handler * dispatcher();
    static void incoming(CoapPacket * in);
    static void outgoing(CoapPacket * out);
    
    private:
    CoapTransport();
    ~CoapTransport();
    void addNonConfirmed(CoapPacket* p);
    static void dispatch();
    CoapSocket * _socket;
    Queue<CoapPacket> * _queue;
    Function_Handler * _handler;
    static CoapTransport * getInstance();
    static CoapTransport * _instance;
};
__END_SYS


#include <utility/random.h>
#include <utility/string.h>
#include <coap_packet.h>
#include <coap_transport.h>

__BEGIN_SYS

/*
 * For a new Confirmable message, the initial timeout is set
 * to a random duration (often not an integral number of seconds)
 * between ACK_TIMEOUT and (ACK_TIMEOUT * ACK_RANDOM_FACTOR) (see
 * Section 4.8)
 * */
CoapPacket::CoapPacket(const UDP_Address & to) : CoapPDU(_pduBuffer, _maxPDUSize, 4),
     _remote(to)
{
    setVersion(1);
    _isConfirmed = false;
    int range = (CoapPacket::ackTimeout * CoapPacket::ackRandomFactor) - CoapPacket::ackTimeout;
    int rand = Pseudo_Random::random() % range;
    setMessageID(rand);
}

CoapPacket::CoapPacket() : CoapPDU(_pduBuffer, _maxPDUSize, 4), _remote(UDP_Address("10.0.2.15:5863"))
{
    setVersion(1);
    _isConfirmed = false;
    int range = (CoapPacket::ackTimeout * CoapPacket::ackRandomFactor) - CoapPacket::ackTimeout;
    int rand = Pseudo_Random::random() % range;
    setMessageID(rand);
}

CoapPacket::CoapPacket(const UDP_Address & from, const char * data, unsigned len) : CoapPDU(_pduBuffer, _maxPDUSize, len),
    _remote(from)
{
    strncpy((char*)_pduBuffer, data, len);
}

CoapPacket::CoapPacket(CoapPDU * pdu) : CoapPDU(pdu->getPDUPointer(), _maxPDUSize, pdu->getPDULength()),
    _remote(UDP_Address("10.0.2.15:5863"))
{
    setMessageID(pdu->getMessageID());
}

void CoapPacket::reset()
{
    CoapPDU::reset();
    _isConfirmed = false;
}

int CoapPacket::getToken()
{
    return atol((char*)getTokenPointer());
}

CoapPacket::~CoapPacket()
{
}

CoapConfirmable::~CoapConfirmable()
{
    if (_alarm) delete _alarm;
    _alarm = 0;
}


void CoapConfirmable::reset()
{
    CoapPacket::reset();
    if (_alarm) delete _alarm;
    _alarm = 0;
}

void CoapPacket::setConfirmed()
{
    _isConfirmed = true;
}

bool CoapPacket::isConfirmed()
{
    return _isConfirmed;
}

UDP_Address CoapPacket::remote() const
{
    return _remote;
}

bool CoapPacket::isConfirmable()
{
    return (getType() == CoapPacket::COAP_CONFIRMABLE);
}

void CoapPacket::generateNewToken(unsigned int len)
{
    if (len > 4) return;
    kout << "size of token: " << len << endl;
    int range = 0xFF << len;
    kout << "range:" << range << endl;
    int rand = Pseudo_Random::random() % range;
    kout << "random:" << rand << endl;
    static const int size = 20;
    char buf[size];
    itoa(rand, buf);
    setToken((u8*)buf, len);
}

void CoapConfirmable::update()
{
    kout << "Confirmable update" << endl;
    if (!isConfirmable()) return;
    _retransmissionCounter++;
    _timeout = _timeout * 2;
    if (_alarm) delete _alarm;
    if (isConfirmed()) {
        kout << "already confirmed!" << endl;
        return;
    }
    kout << "configure new alarm with " << _timeout << endl;
    _alarm = new Alarm(_timeout, CoapTransport::dispatcher(), 1);
}

bool CoapConfirmable::isFailure()
{
    return (_retransmissionCounter > CoapPacket::maxRetransmit);
}
        
CoapConfirmable::CoapConfirmable(CoapPacket::Code code) : CoapPacket() {
    setType(CoapPacket::COAP_CONFIRMABLE);
    setCode(code);
    int range = (CoapPacket::ackTimeout * CoapPacket::ackRandomFactor) - CoapPacket::ackTimeout;
    int rand = Pseudo_Random::random() % range;
    _timeout = CoapPacket::ackTimeout + rand;
    _retransmissionCounter = 0;
    _alarm = 0;
}


__END_SYS


#include <system/config.h>
#include <coap_request.h>
#include <coap_transport.h>

__BEGIN_SYS

CoapRequest * CoapRequest::_pendingRequests[] = {0};

CoapRequest::CoapRequest(CoapPacket::Code code, const char* uri)
    : CoapConfirmable(code)
{
    _uriSize = strlen(uri);
    strncpy(_uriBuffer, uri, _uriSize);
    generateNewToken(4);
    setURI(_uriBuffer, _uriSize);
    CoapTransport::outgoing(this);
    addAsPending();
}

void CoapRequest::incomingResponse(CoapPacket * packet)
{
    CoapRequest * req = removePendingByToken(packet->getToken());
    if (req) req->addResponse(packet);
}

void CoapRequest::addResponse(CoapPacket * packet)
{
    if (packet->isFailure()) {
        onError();
    } else {
        onSuccess();
        _response = packet;
    }
}

void CoapRequest::addAsPending()
{
    unsigned int index = indexByToken(getToken());
    CoapRequest::_pendingRequests[index] = this;
}

unsigned int CoapRequest::indexByToken(int token)
{
    return token % CoapRequest::_maxPending;
}

CoapRequest * CoapRequest::removePendingByToken(int token)
{
    unsigned int index = indexByToken(token);
    CoapRequest * ptr = CoapRequest::_pendingRequests[index];
    CoapRequest::_pendingRequests[index] = 0;
    return ptr;
}

void CoapRequest::onSuccess()
{
    kout << "success!" << endl;
}

void CoapRequest::onError()
{
    kout << "error!" << endl;
}

__END_SYS


#include <alarm.h>
#include <utility/handler.h>
#include <utility/list.h>
#include <udp.h>
#include <coap_request.h>
#include <coap_transport.h>

__BEGIN_SYS

static const int idsSize = 1000;
static int ids[idsSize] = {0};

CoapTransport * CoapTransport::_instance = 0; 

static void updateIds(int id) {
    int pos = id % idsSize;
    ids[pos] = id;
}

static bool checkId(int id) {
    int pos = id % idsSize;
    return (ids[pos] == id);
}

CoapTransport * CoapTransport::getInstance()
{
    if (CoapTransport::_instance == 0) {
        CoapTransport::_instance = new CoapTransport();
    }
    return CoapTransport::_instance;
}

CoapTransport::CoapTransport()
{
    _handler = new Function_Handler(CoapTransport::dispatch);
    _queue = new Queue<CoapPacket>();
    _socket = new CoapSocket(UDP_Address(IP::instance()->address(), COAP_PORT));
}

void CoapTransport::dispatch()
{
    CoapTransport * t = CoapTransport::getInstance();
    if (!t) return;
    Queue<CoapPacket>::Element * e = t->_queue->remove();
    CoapPacket * p = e->object();
    if (!p) return;
    if (p->isFailure()){
        kout << "failure "<< endl;
        if (e) delete e;
        return;
    }
    outgoing(p);
    if (e) delete e;
}

void CoapTransport::addNonConfirmed(CoapPacket * p)
{
    Queue<CoapPacket>::Element * e = new Queue<CoapPacket>::Element(p);
    _queue->insert(e);
}

CoapTransport::~CoapTransport()
{
    delete _handler;
    delete _socket;
    delete[] _queue;
}

void CoapTransport::incoming(CoapPacket * in) {
    if (!in) return;
    CoapTransport * t = CoapTransport::getInstance();
    if (!t) return;
    kout << "INCOMING packet" << endl;
    in->print();
    if (in->getType() == CoapPacket::COAP_ACKNOWLEDGEMENT) {
        kout << "is ACK!" << endl;
        updateIds(in->getMessageID());
    }
    if (in->getType() >= CoapPacket::COAP_CREATED) {
        kout << "is a Response!" << endl;
        CoapRequest::incomingResponse(in);
    }
}

void CoapTransport::outgoing(CoapPacket * out)
{
    if (!out) return;
    kout << "OUTGOING packet" << endl;
    kout << "-------------------------" << endl;
    out->print();
    kout << "-------------------------" << endl;
    CoapTransport * t = CoapTransport::getInstance();
    if (!t) return;
    t->_socket->sendPacket(out);
    if (out->isConfirmable()) {
        kout << "Is confirmable!" << endl;
        if (checkId(out->getMessageID())) {
            kout << "is confirmed!" << endl;
            return;
        }
        t->addNonConfirmed(out);
        out->update();
    }
}

Function_Handler * CoapTransport::dispatcher()
{
    CoapTransport * t = CoapTransport::getInstance();
    if (!t) return 0;
    return _instance->_handler;
}

__END_SYS


#include <coap_transport.h>
#include <coap_socket.h>

__BEGIN_SYS

CoapSocket::CoapSocket(const UDP_Address & local) : UDP::Socket(local, UDP::Address(Traits<IP>::BROADCAST, COAP_PORT))
{
}

CoapSocket::~CoapSocket()
{
}

void CoapSocket::received(const UDP_Address& from, const char* data, unsigned int len)
{
	CoapPacket *packet = new CoapPacket(from, data, len);

	if(!packet->validate()) return;
	
	packet->print();

	if (packet->getType() == CoapPacket::COAP_ACKNOWLEDGEMENT) {
	    kout << "is ack!" << endl;
    }

    CoapTransport::incoming(packet);
}

void CoapSocket::sendPacket(CoapPacket * pdu)
{
    if (!pdu) return;
    remote(pdu->remote());
    send(reinterpret_cast<char*>(pdu->getPDUPointer()), pdu->getPDULength());
}

__END_SYS
\end{lstlisting}


\subsection{Aplica\c{c}\~ao WEB}

\end{document}
