%%%%%%%%%%%%%%%%%%%%%%%%%%%%%%%%%%%%%%%%%%%%%%%%%%%%%%%%%%%%%%%%%%%%%%
% Universidade Federal de Santa Catarina             
% Biblioteca Universitária                     
%----------------------------------------------------------------------
% Exemplo de utiliza\c{C}ão da documentclass ufscThesis
%----------------------------------------------------------------------
% (c)2013 Roberto Simoni (roberto.emc@gmail.com)
%         Carlos R Rocha (cticarlo@gmail.com)
%         Rafael M Casali (rafaelmcasali@yahoo.com.br)
%%%%%%%%%%%%%%%%%%%%%%%%%%%%%%%%%%%%%%%%%%%%%%%%%%%%%%%%%%%%%%%%%%%%%%%
\documentclass{ufscThesis} % Definicao do documentclass ufscThesis	

%----------------------------------------------------------------------
% Pacotes usados especificamente neste documento
\usepackage{graphicx} % Possibilita o uso de figuras e gráficos
\usepackage{color}    % Possibilita o uso de cores no documento
\usepackage{listings}
\usepackage{pgfgantt}
\usepackage{enumitem}
\usepackage{amsmath}
\usepackage{paralistt
%----------------------------------------------------------------------
% Comandos criados pelo usuário
\newcommand{\afazer}[1]{{\color{red}{#1}}} % Para destacar uma parte a ser trabalhada

%----------------------------------------------------------------------
% Identificadores do trabalho
% Usados para preencher os elementos pré-textuais
\instituicao[a]{Universidade Federal de Santa Catarina} % Opcional
\departamento[a]{Departamento de Inform\'atica e Estat\'istica}
\grau{Bacharel em Ci\^encias da Computa\c{c}\~ao}
\curso[o]{Ci\^encias da Computa\c{c}\~ao}
\documento[o]{Trabalho de conclus\~ao de curso} % [o] para disserta\c{C}ão [a] para tese
\titulo{Implementa\c{c}\~ao do protocolo CoAP para servi\c{c}os de monitoramento em redes de sensores sem fio}
\autor{Rafael de Lucena Valle}
\local{Florian\'opolis} % Opcional (Florianópolis é o padrão)
\data{15}{julho}{2013}
\orientador[Orientador]{Prof. Dr. Ant\^onio Augusto Fr\"ohlich}
\coorientador[Coorientador]{Prof. M.Sc. Arliones Hoeller Jr}
\coordenador[Coordenador]{Prof. Dr. Vit\'orio Mazzola}

\numerodemembrosnabanca{3} % Isso decide se haverá uma folha adicional
\orientadornabanca{sim} % Se faz parte da banca definir como sim
\coorientadornabanca{nao} % Se faz parte da banca definir como sim
\bancaMembroA{Prof. Dr. Ant\^onio Augusto Fr\"ohlich} %Nome do presidente da banca
\bancaMembroB{Prof. Dr. Eng. Rafael Luiz Cancian} % Nome do membro da Banca
\bancaMembroC{Prof. Dr. Frank Siqueira} % Nome do membro da Banca

\agradecimento{Inserir os agradecimentos aos colaboradores \'a execu\c{c}\~ao do trabalho.}

\textoResumo {  Redes de sensores s\~ao utilizadas para a capta\c{c}\~ao, processamento de informa\c{c}\~ao\
    e atua\c{c}\~ao sobre um ambiente, tornando-as importantes em aplica\c{c}\~oes de controle, telemetria \
    e rastreamento.
    
    Os n\'os destas redes possuem processadores, transmissores e receptores de dados simplificados, que possuem restri\c{c}\~oes\
    de mem\'oria, processamento, energia e taxa de transmiss\~ao de dados. Contudo estes dispositivos apresentam um custo\
    baixo de equipamentos, tornando mais interessante a implanta\c{c}\~ao destes sistemas.
    
    O protocolo HTTP, um protocolo de aplica\c{c}\~ao muito utilizado na atualidade, foi desenvolvido pensado\
    em computadores de prop\'osito geral, onde essas restri\c{c}\~oes n\~ao existem. Um protocolo leve como\
    CoAP pode tornar vi\'avel a cria\c{c}\~ao de aplica\c{c}\~oes web em redes de sensores sem fio por um baixo custo. Al\'em disso o CoAP implementa mecanismos básicos de um SOA.
    
    Este trabalho prop\~oe uma infraestrutura de comunica\c{c}\~ao entre redes de sensores sem fio e a Internet,\
utilizando protocolos leves entre os n\'os sensores e um gateway GPRS para \'areas sem acesso \`a WIFI,\
aproveitando a vasta abrang\^encia da tecnologia de telefonia. Com a utiliza\c{c}\~ao do CoAP \'e esperado uma redu\c{c}\~ao\
de consumo de energia e mem\'oria, em rela\c{c}\~ao a outros protocolos de aplica\c{c}\~ao existentes.}

\palavrasChave{ internetworking WSN IPv6 6LoWPAN GPRS CoAP IoT}

%----------------------------------------------------------------------
% Início do documento                                
\begin{document}
%--------------------------------------------------------
% Elementos pré-textuais
\capa
\folhaderosto%[comficha] % Se nao quiser imprimir a ficha, é só não usar o parâmetro
%\folhaaprovacao
%\paginadedicatoria
%\paginaagradecimento
%\paginaepigrafe

\paginaresumo
%\paginaabstract
%\pretextuais % Substitui todos os elementos pre-textuais acima
%\listadefiguras % as listas dependem da necessidade do usuário
%\listadetabelas 
\abreviatura{CoAP}{Constrained Aplication Protocol}
\abreviatura{EPOS}{Embedded Parallel Operating System}
\abreviatura{HTTP}{Hipertext Transfer protocol}
\abreviatura{IETF}{Internet Engineering Task Force}
\abreviatura{M2M}{Machine-to-Machine}
\abreviatura{REST}{Representatational State Transfer}
\abreviatura{UDP}{User Datagram Protocol}
\abreviatura{6LoWPan}{IPv6 over Low power Wireless Personal Area Networks}
\abreviatura{IoT}{Internet of Things}
\abreviatura{JSON}{JavaScript Object Notation}
\abreviatura{XML}{Extensible Markup Language}
\abreviatura{ADESD}{Application Driven Embedded System Design}

\listadeabreviaturas

%\listadesimbolos
\sumario
%--------------------------------------------------------
% Elementos textuais

\chapter{Introdu\c{c}\~ao}
Redes de sensores s\~ao utilizadas para a capta\c{c}\~ao, processamento de informa\c{c}\~ao e atua\c{c}\~ao sobre um ambiente, tornando-as importantes em aplica\c{c}\~oes de controle, telemetria e rastreamento de sistemas.

N\'os que participam destas redes geralmente s\~ao computadores e r\'adios simplificados, que possuem restri\c{c}\~oes de mem\'oria, processamento, energia e capacidade de comunica\c{c}\~ao, mas um custo relativamente baixo de equipamentos.

O maior consumo de energia neste tipo de aplica\c{c}\~ao \'e do r\'adio, portanto os desafios dos algoritmos de comunica\c{c}\~ao nesta \'area s\~ao manter os r\'adios ligados o m\'inimo de tempo poss\'ivel sem comprometer a conectividade do n\'o.

\section{Objetivos}
O objetivo geral desse trabalho \'e descrever e implementar webservices em uma rede sensores sem fio, que far\~ao\
a aquisi\c{c}\~ao dos dados do ambiente e disponibilizar\~ao as informa\c{c}\~oes captadas na Internet.

\subsection{Objetivos Espec\'ificos}
Os objetivos espec\'ificos do trabalho s\~ao relacionados ao desenvolvimento de duas aplica\c{c}\~oes e uma biblioteca.

%TODO Melhorar!

\begin{itemize}
    \item Portar o protocolo CoAP, Constrained Application Protocol para o EPOS, Embedded Paralell Operating System.
    \item Implementar uma aplica\c{c}\~ao de redes de sensores sem fio.
    \item Desenvolver o firmware do  gateway GPRS/Zigbee, que ir\'a disponibilizar as informa\c{c}\~oes captadas da rede para Internet.
\end{itemize}
        

\section{Justificativa}

Os mecanismos de confiabilidade na transmiss\~ao de dados, t\'ecnicas para se manter uma conex\~ao do TCP e rearranjos que s\~ao feitos para garantir a ordem das mensagens recebidas n\~ao s\~ao adequados para um dispositivo que dependam de uma bateria, pois podem fazer que fiquem com seus transmissores, ligados por mais tempo para manter a conex\~ao ou at\'e mesmo para reenvio de mensagens. O maior consumo de energia de um n\'o sensor \'e no envio e recebimento de dados, quando mantem seu transmissor ligado.

Assim faz-se uso do UDP, um protocolo que n\~ao mant\'em conex\~ao onde os dados s\~ao recebidos fora de ordem e o envio \'e feito de uma mensagem por vez, sem o uso de streammings do TCP, que fazem que quem receba a mensagem precise mont\'a-la e garantir que nenhuma das pe\c{c}as est\'a corrompida. Isto implica tamb\'em na redu\c{c}\~ao do tamanho do cabe\c{c}alho do pacote. Estas caracter\'isticas demostram uma alternativa interessante para estes equipamentos limitados. Testes feitos em implementa\c{c}\~oes de sistemas operacionais similares ao EPOS, como Contiki e TinyOS, utilizando o protocolo CoAP demonstram redu\c{c}\~ao no consumo de energia e mem\'oria em rela\c{c}\~ao ao HTTP.\cite{}

A falta de padroniza\c{c}\~ao dos protocolos afeta o desenvolvimento de uma rede p\'ublica ub\'iqua de uma cidade inteligente por exemplo. Grande parte das solu\c{c}\~oes utiliza protocolos propriet\'arios, que se comunicacam apenas com os produtos de um mesmo fabricante.

O protocolo HTTP foi desenvolvido para comunica\c{c}\~ao de computadores de prop\'osito geral, onde as restri\c{c}\~oes citadas n\~ao s\~ao comuns. Em rela\c{c}\~ao ao tamanho, o pacote HTTP \'e um problema para redes ZigBEE, j\'a que estas redes que possuem uma restri\c{c}\~ao de 128 bytes. O protocolo TCP precisa transmitir mensagens adicionais para manter uma conex\~ao, outra caracter\'istica que n\~ao \'e interessante para RSSF.

Um protocolo leve como CoAP pode tornar vi\'avel a cria\c{c}\~ao de aplica\-\c{c}\~oes web em redes de sensores sem fio por um baixo custo. Neste trabalho \'e proposto uma infraestrutura de comunica\c{c}\~ao entre redes de sensores sem fio e a Internet, utilizando protocolos leves entre os n\'os sensores e um gateway GPRS para \'areas sem acesso \`a WIFI, aproveitando a vasta abrang\^encia da tecnologia de telefonia. Com a utiliza\c{c}\~ao do CoAP \'e esperado uma redu\c{c}\~ao de consumo de energia e mem\'oria, em rela\c{c}\~ao a outros protocolos de aplica\c{c}\~ao existentes.

Em lugares aonde n\~ao existe o acesso a rede cabeada ou sem fio, como lugares afastados, na \'area rural, por exemplo a distribui\c{c}\~ao da informa\c{c}\~ao para Internet ser\'a feita atrav\'es de um gateway. Este gateway ser\'a composto por um EposMoteII e um m\'odulo GPRS.
Atualmente o padr\~ao GPRS oferece a maior cobertura dentre as tecnologias de transmiss\~ao de telefonia no Brasil, atingindo cerca de 5477 munic\'ipios.

\section{Metodologia}

Ser\'a feito um levantamento dos componentes necess\'arios para o desenvolvimento do gateway 6LowPan/GPRS utilizando o mote EPOSMote II e um modem GPRS, que ser\'a definido no decorrer do trabalho. Ent\~ao ser\'a desenvolvido o esquem\'atico para fabrica\c{c}\~ao do gateway.

Ap\'os a valida\c{c}\~ao do modelo, ser\'a iniciada a prototipa\c{c}\~ao do hardware e a implementa\c{c}\~ao do protocolo de aplica\c{c}\~ao CoAP no EPOS. O desenvolvimento do protocolo ser\'a orientado a testes, aonde o c\'odigo escrito apenas satisfaz as condi\c{c}\~oes necess\'arias para validar um comportamento desejado da aplica\c{c}\~ao.

Nos testes de integra\c{c}\~ao do gateway, ser\'a utilizada uma placa de desenvolvimento em conjunto com um m\'odulo M95 da Quectel disponibilizada pelo Laborat\'orio, para serem feitos os testes de envio de mensagens em diversos protocolos, inclusive testes com comandos propriet\'arios adicionais do modem.

Para testes de integra\c{c}\~ao, as aplica\c{c}\~oes ser\~ao executados na plataforma de sensores sem fio EPOS Mote II utilizando o EPOS com o CoAP desenvolvido.


\chapter{Revis\~ao Bibliografica}
Avan\c{c}os recentes nas tecnologias de sistemas eletr\^onicos, semicondutores, sensores, microcontroladores e r\'adios tornaram poss\'ivel o desenvolvimento de redes de sensores de baixo custo e baixo consumo uma realidade.

Este cap\'itulo apresenta uma vis\~ao geral sobre redes de sensores sem fio, arquitetura orientada a servi\c{c}os e os protocolos de aplica\c{c}ao existentes.

\section{Redes de sensores sem fio}

Redes de sensores sem fio s\~ao utilizados para capturar, processar e comunicar dados captados do ambiente. Geralmente tais redes possuem centenas ou milhares de sensores e possuem as seguintes caracter\'isticas: pouca mem\'oria, pouco alcance do r\'adio, baixa capacidade de processamento e bateria, e custo reduzido.

Um n\'o pertencente a esta rede geralmente \'e um dispositivo especificamente desenvolvido para um pr\'oposito, que possui poucos recursos computacionais e energ\'eticos e se comunicam entre seus semelhantes, como apresentado na figura \ref{wsnOverview}.
\begin{figure}[h]
   \label{wsnOverview}
   \centering
   \includegraphics[width=0.7\textwidth]{figuras/wsn.png}
   \caption{Overview de uma redes de sensores sem fio}
\end{figure}

A conserva\c{c}\~ao de energia \'e um dos objetivos das redes de sensores sem fio, pois n\~ao est\~ao ligados diretamente a fonte de energia. Deve-se minimizar o consumo em todos os n\'iveis do sistema, da aplica\c{c}\~ao at\'e o meio f\'isico, iniciando com o projeto de r\'adio. \cite{WsnSurvey2008} 


\section{Arquitetura orientada a servi\c{c}os}

Arquitetura orientada a servi\c{c}os \'e uma forma de organizar infraestrutura e aplica\c{c}\~oes de software em um conjunto de servi\c{c}os. Estes s\~ao oferecidos por prestadores de servi\c{c}o, servidores, organiza\c{c}\~oes que implementam os servi\c{c}os, fornecem descri\c{c}\~ao dos servic\c{c}os oferecidos, suporte t\'ecnico e de neg\'ocio.

O modelo de computa\c{c}\~ao utilizando esta afirmativa \'e conhecido como Computa\c{c}\~ao Orientada a servi\c{c}os (SOC). \cite{581580}

Clientes destes servi\c{c}os podem ser outras solu\c{c}\~oes, aplica\c{c}\~oes, processos ou usu\'arios. Para satisfazer estes requis\'itos servi\c{c}os devem:
\begin{itemize}
\item Tecnologicamente neutros: utilizar-se de padr\~oes reconhecidos e bem aceitos para comunica\c{c}\~ao, descri\c{c}\~ao e mecanismos de descoberta;
\item Baixo acoplamento: detalhes desnecess\'arios (o qu\~ao desnecess\'ario precisa ser discutido) devem ser escondidos do cliente, que n\~ao precisa ter conhecimento sobre o funcionamento interno para utilizar o servi\c{c}o;
\item Localidade transparente: clientes devem ser atendidos independentemente da localidade do servi\c{c}o dispon\'ivel.
\end{itemize}

Abaixo uma figura \ref{soaOverview}.

\begin{figure}[h]
   \label{soaOverview}
   \centering
   \includegraphics[width=0.6\textwidth]{figuras/soa.png}
   \caption{Arquitetura Orientada a Servi\c{c}os}
\end{figure}

\section{RESTful}

Um servi\c{c}o web que utiliza HTTP e os princ\'ipios REST possui recursos e a\c{c}\~oes gen\'ericas bem definidas.\cite{rest}

Para transfe\^encia de dados utiliza-se formatos gen\'ericos que enfatizam simplicidade e usabilidade pela internet, como XML e JSON.

Os Recursos s\~ao usam um identificador \'unico e persistente, as URIs. A URIs possuem estruturas de diret\'orios, uma URI \'e uma \'arvore com ramos subordinados e superordinados conectando os n\'os. As opera\c{c}\~oes suportas s\~ao m\'etodos HTTP expl\'icitos que n\~ao salvam estado das aplica\c{c}\~oes clientes e s\~ao idempotentes, s\~ao eles:
GET: solicita ao webserver a representa\c{c}\~ao de uma informa\c{c}\~ao de um determinado recurso.
POST: criar um recurso no webserver.
PUT: mudar o estado de um recurso do webserver.
DELETE: remover o recurso ou alterar para um estado vazio.

Uma abordagem utilizando SOAP RPC em HTTP n\~ao \'e interessante para uma aplica\c{c}\~ao de RSSF, j\'a que a quantidade de informa\c{c}\~ao a ser transmitida \'e consideravelmente maior. Al\'em disso, a aplica\c{c}\~ao teria que conhecer l\'ogica interna do servi\c{c}o istrumentando o recurso utilizando chamadas de fun\c{c}\~oes remotas. A figura \ref{soaVsHttp} exemplifica e demonstra a diferen\c{c}a de uma aplica\c{c}\~ao que faz uso de SOAP RPC e outra RESTful.\cite{richardson2008restful}

A figura \ref{bytesTransmitted} faz um comparativo entre o n\'umero de bytes transmitidos de diversos servidores web e seus protocolos.

\begin{figure}[h]
    \label{bytesTransmitted}
    \centering
    \includegraphics[width=0.7\textwidth]{figuras/bytestransmitted.png}
    \caption{Imagem retirada de \cite{transportApp}}
\end{figure}




%Inserir figura que demonstra a diferença entre o tamanho dos pacotes.
\afazer{A desenvolver...}


\section{CoAP}

Um dos principais objetivos do CoAP \'e ser uma alternativa protocolo web gen\'erico para redes com dispositivos com restri\c{c}\~ao de energia e mem\'oria.

As vantagens de utilizar um protocolo compat\'ivel com o HTTP s\~ao: a facilidade de integra\c{c}\~ao e o reuso de aplica\c{c}\~oes. CoAP \'e um conjunto REST otimizado para M2M, com suporte a descoberta de recursos, multicast e troca de mensagens ass\'incronas com simplicidade e baixo overhead.

A IETF estabelece as condi\c{c}\~oes m\'inimas para o desenvolvimento de um protocolo de aplica\c{c}\~ao compat\'ivel com HTTP, mas focado em aplica\c{c}\~oes aonde energia e mem\'oria s\~ao escassas. O protocolo CoAP foi projetado levando em considera\c{c}\~ao as restri\c{c}\~oes energ\'eticas e altas taxas de falha na transmiss\~ao dos pacotes em RSSF.

A comunica\c{c}\~ao entre os pontos no CoAP \'e de forma ass\'incrona usando o UDP. A confiabilidade \'e um par\^ametro opcional e funciona atrav\'es de um mecanismo de retransmiss\~ao exponencial. Possui 4 tipos de mensagem: Confirm\'avel, N\~ao-Confirm\'avel, Confirma\c{c}\~ao (ACK) e Reset. A figura \ref{coapFormat} mostra o formato do pacote.

\subsection{Formato das mensagens}
Uma mensagem CoAP deve caber num \'unico pacote IP, para que seja transmitida numa camada de enlace limitada.
\begin{figure}[h]
    \label{coapFormat}
    \centering
    \includegraphics[width=0.7\textwidth]{figuras/formato.png}
    \caption{formato do pacote CoAP  \cite{draft-ietf-core-coap-18}}
\end{figure}


Os campos do pacote CoAP s\~ao: a vers\~ao do CoAP, implementa\c{c}\~oes devem utilizar este campo com o valor 1. O tipo: campo para definir o tipo da mensagem: Confirm\'avel (0), N\~ao-Confirm\'avel (1) , de Confirma\c{c}\~ao (2) ou Reset (3).

O tamanho do Token: utilizado para controle de requisi\c{c}\~oes e repostas. O tamanho do Token pode variar entre 0 e 8 bytes. Tamanhos entre 9 a 15 s\~ao reservados e n\~ao devem ser usados. \'E um campo sempre gerado pelo cliente CoAP.

O C\'odigo: separados em 3-bit mais significativos para classes e 5-bits menos significativos para detalhe. As classes podem indicar uma requisi\c{c}\~ao (0), uma resposta de sucesso (2), e uma resposta de erro do cliente (4), ou uma resposta de erro do servidor (5), as outras classes s\~ao reservadas. Em um caso especial o c\'odigo 0.00 indica uma mensagem vazia.

O ID da mensagem: usada para deduplica\c{c}\~ao de mensagens e confirma\c{c}\~ao ou reset de mensagens. \'E gerado por quem envia a mensagem, no caso de uma mensagem confirm\'avel ou reset, a resposta deve possuir o ID da mensagem enviada. A implemeta\c{c}\~ao da gera\c{c}\~ao dos IDs est\'a aberta, depende da aplica\c{c}\~ao que o CoAP ser\'a usado, por\'em \'e recomendado que o valor inicial seja rand\^omico.
   
%Codifica\c{c}\~ao das op\c{c}\~oes
\subsection{Transmiss\~ao de Mensagens}
A transmiss\~ao de mensagems \'e controlada basicamente pelos par\^ametros: ACK TIMEOUT, ACK RANDOM FACTOR, MAX RETRANSMIT, NSTART, Leisure e PROBING RATE.

Estes par\^ametros s\~ao respectivamente: o tempo que uma mensagem confirm\'avel aguarda o ACK; fator de randomicidade para gerar os ACK TIMEOUTs subsequentes; contador para o n\'umero m\'aximo de tentativas de retransmiss\~ao; n\'umero limite de intera\c{c}\~oes simult\^aneas mantidas por um servidor.

A Leisure \'e o tempo que o servidor aguarda para responder uma requisi\c{c}\~ao multicast, \'e calculada: $Leisure = S * G / R$. Aonde S \'e o tamanho estimado da reposta, G \'e uma estimativa do tamanho do grupo e R \'e a taxa de transmiss\~ao. PROBING RATE: \'e a taxa m\'edia para transmiss\~ao de dados.

    Estes par\^ametros definem a temporiza\c{c}\~ao do sistema. Os valores padr\~oes s\~ao mostrados na Tabela \ref{coapDefault}.
\begin{table}[h]
\label{coapDefault}
\centering
\begin{tabular}{@{}lllll@{}}
\toprule
Nome & Valor padr\~ao & \\ \midrule
ACK timeout & 2 segundos & \\
ACK random factor & 1.5 & \\
NStart & 1 & \\
Default Leisure & 5 segundos & \\
Probing rate & 1 Byte/segundo & \\
Max retransmit & 4 &  \\ \midrule
\end{tabular}
\caption{Valores padr\~ao do CoAP.}
\end{table}

A retransmiss\~ao \'e controlada por um timeout e um contador. Quando este timeout \'e atigido e o contador \'e menor que valor m\'aximo de retransmiss\~ao a mensagem \'e transmitida, o contador incrementado e timeout duplicado. O modelo de retransmiss\~ao usa um contador de timeouts e uma fun\c{c}\~ao que varia de acordo com o n\'umero de tentativas.

Uma falha na transmiss\~ao ocorre quando atingir o n\'umero m\'aximo de tentavivas ou receber uma mensagem de RESET. Quando receber um ACK a transmiss\~ao da mensagem confirm\'avel \'e completa.

O servidor ir\'a ignorar mensagens que chegam por multicast quando n\~ao puder responder nada de \'util.

Na situa\c{c}\~ao aonde possuir uma informa\c{c}\~ao suficientemente nova pode responder na pr\'opria mensagem de confirma\c{c}\~ao (ACK). Essa t\'ecnica \'e chamada de ''Piggy-backed'' um mecanismo de transmiss\~ao para mensagens confirmadas, o cen\'ario \'e ilustrado na Figura \ref{piggyBacked}.\cite{draft-ietf-core-coap-18}
\begin{figure}[h]
   \label{piggyBacked}
   \centering
   \includegraphics[width=0.6\textwidth]{figuras/piggybacked.png}
   \caption{Resposta na mensagem de confirma\c{c}\~ao, chamado de piggy-backed.}
\end{figure}

Fluxo esperado de requisi\c{c}\~ao sem confirma\c{c}\~ao na figura \ref{nonConfirmable}.
\begin{figure}[h]
   \label{nonConfirmable}
   \centering
   \includegraphics[width=0.6\textwidth]{figuras/nonconfirmable.png}
   \caption{Fluxo esperado de requisi\c{c}\~ao e resposta sem confirma\c{c}\~ao}
\end{figure}

A RFC tamb\'em prev\^e fluxo de requisi\c{c}\~ao com confirma\c{c}\~ao, e resposta separada com confirma\c{c}\~ao. A figura \ref{separateResponse} exemplifica:

\begin{figure}[h]
   \label{separateResponse}
   \centering
   \includegraphics[width=0.6\textwidth]{figuras/separateresponse.png}
   \caption{Fluxo esperado de requisi\c{c}\~ao e resposta com confirma\c{c}\~ao, com resposta separada}
\end{figure}

\subsection{Camada de Reposta e Requisi\c{c}\~ao}
Uma requisi\c{c}\~ao \'e inicializada ao preencher o campo code no cabe\c{c}alho do CoAP. Possuem as mesmas propriedades de idempot\'encia e only retrieved das requisi\c{c}\~oes HTTP.

\subsection{Recursos}
A descoberta de recursos \'e feita quando um servidor recebe uma requisi\c{c}\~ao GET para o recurso ~/well-know/core. O servidor CoAP deve responder no formato CORE link Format.\cite{rfc6690} E a descoberta de servi\c{c}os no protocolo CoAP \'e feita atrav\'es de socket Multicast. Os recursos s\~ao identificados por uma URI, e os m\'etodos s\~ao implementados de forma similar ao HTTP.

\afazer{A desenvolver...}

%Inserir figura que represente os recursos ???

\section{EPOS}
O EPOS \'e um sistema operacional multithread com suporte a preemp\c{c}\~ao, foi desenvolvido em C++ que faz uso intenso de programa\c{c}\~ao orientada a aspectos utilizando templates.

Possui abstra\c{c}\~oes para entidades temporais como rel\'ogio, alarme e cron\^ometro, biblioteca com estruturas de dados e sequenciadores. Permitindo o uso de ferramentas para gera\c{c}\~ao automatizada de abstra\c{c}\~oes de sistemas. A portabilidade \'e atingida utilizando entidades chamados de Mediadores de Hardware que fornecem interfaces simples para acesso as fun\c{c}\~oes espec\'ificas de arquitetura. Estas interfaces s\~ao utilizadas por entidades abstratas como alarmes e threads peri\'odicas. A figura 5 mostra uma vis\~ao abstrata da arquitetura do EPOS.

%Inserir figura da arquitetura do EPOS.

Foi projetado utilizando ADESD, Application Driven Embedded System Design, um m\'etodo para projeto de sistemas embarcados orientados \`a aplica\c{c}\~ao. Esta metodologia guia o desenvolvimento paralelo de hardware e software al\'em de manter portabilidade. O EPOS possui porte para as seguintes arquiteturas: MIPS, IA32, PowerPC, H8, Sparc, AVR e ARM. \cite{epos}

Utiliza um sistema de constru\c{c}\~ao baseado em makefiles e shell scripts.

\section{Trabalhos Relacionados}

\subsection{Contiki}
O Contiki \'e um sistema operacional criado por Adam Dunkels em 2000, escrito em C, de c\'odigo aberto para sistemas com restri\c{c}\~ao de recursos comunicam-se numa rede. Foi desenvolvido para ser um sistema operacional para Internet das coisas. Possui uma camada de abstra\c{c}\~ao RESTful para web services chamada Erbium, que implementa o protocolo CoAP.

Cada processo no Contiki possui bloco de controle, que cont\'em informa\-\c{c}\~oes de tempo de execu\c{c}\~ao do processo e uma refer\^encia para uma protothread, na qual o c\'ogido \'e armazenado na ROM. 

Protothread \'e uma combina\c{c}\~ao entre eventos e threads, possuem comportamentos de bloqueio e espera, que permite o intersequenciamento dos eventos, gerando um baixo overhead de mem\'oria por n\~ao necessitar de salvamento de contexto.

Cada protothread consome 2 bytes de mem\'oria, que s\~ao utilizados para armazenar a continuidade local, uma referencia utilizada em um pulo condicional durante a execu\c{c}\~ao da thread. \'E um m\'etodo similar ao mecanismo de Duffy e Co-rotina em C. \cite{duffy}

O transceiver sem fio \'e um dos componentes que mais consome energia quando ligado escutando o ambiente, assim uma das estrat\'egias utilizadas \'e manter o m\'inimo de tempo poss\'ivel ligado, mas o suficiente para manter a troca de mensagens na rede. O Contiki prop\~oe uma estrat\'egia de ciclos de trabalho que consegue manter um n\'o comunic\'avel em uma rede, por\'em com seus r\'adios desligados em aproximadamente 99\% do tempo.

\subsection{LibCoap}
LibCoap \'e uma biblioteca implementada em C do protocolo CoAP. Possui 292K de tamanho compilada estaticamente em sua vers\~ao 4.0.1.
A licensa da biblioteca \'e GPL (2 ou maior) ou licensa BSD revisada.

Possui uma su\'ite de testes para regress\~ao, utilizando o framework de testes CUnit (http://cunit.sourceforge.net/). A documenta\c{c}\~ao pode ser encontrada em: http://libcoap.sourceforge.net/.

\'E uma biblioteca auto-condida, que possui parser do protocolo e fun\c{c}\~oes b\'asicas de rede utilizando sockets tipo BSD e malloc. Implementa\c{c}\~ao de Hash, String e URI: os headers s\~ao hashkey.h, str.h, uri.h utilizados para montar os pacotes CoAP.

\'E separada num m\'odulo de rede: net.h, aonde \'e implementado as fun\c{c}\~oes de envio/recebimento de requisi\c{c}\~oes e respostas, com confirma\c{c}\~ao, sem confirma\c\~ao, mensagem de reset e erros.

Para selecionar a camada de transporte \'e necess\'ario selecionar utilizando flags de preprocessamento. O padr\~ao \'e socket POSIX. A pilha uIP \'e selecionada com a flag -DWITH\textunderscore CONTIKI, ou para selecionar a pilha lwIP -DWITH\textunderscore LWIP.

\subsection{TinyOS}
O TinyOS \'e um sistema operacional projetado para sistemas embarcados com comunica\c{c}\~ao sem fio e restri\c{c}\~oes energ\'eticas. Foi desenvolvido em nesC, uma linguagem c\'odigo aberto que \'e uma extens\~ao do C. \'E um sistema operacional baseado em eventos desenvolvido para redes de sensores que possuem recursos limitados. Possui uma implementa\c{c}\~ao do CoAP baseada na libCoAP.

\subsection{CantCoap}
\'E uma implementa\c{c}\~ao em C++ que visa facilitar a cria\c{c}\~ao de pacotes Coap tanto diretamente quanto a partir de uma sequ\^encia de characteres, recebida da camada UDP por exemplo.

\'E poss\'ivel montar os pacotes CoAP a partir de uma sequ\^encia de caracteres recebidos de uma placa de rede. Abaixo exemplos de uso da biblioteca retirados de https://github.com/staropram/cantcoap.

\lstdefinestyle{customc}{
  belowcaptionskip=1\baselineskip,
  breaklines=true,
  xleftmargin=\parindent,
  language=C++,
  showstringspaces=false,
  basicstyle=\scriptsize\ttfamily,
  keywordstyle=\bfseries\color{green!40!black},
  commentstyle=\itshape\color{blue},
  identifierstyle=\color{blue!20!black},
  stringstyle=\color{red},
}

\lstset{escapechar=@,style=customc}

Abaixo um exemplo de uso para montar um pacote e enviar:

\begin{lstlisting}
CoapPDU *pdu = new CoapPDU();
pdu->setType(CoapPDU::COAP_CONFIRMABLE);
pdu->setCode(CoapPDU::COAP_GET);
pdu->setToken((uint8_t*)"\3\2\1\0",4);
pdu->setMessageID(0x0005);
pdu->setURI((char*)"test",4);

// send packet 
ret = send(sockfd,pdu->getPDUPointer(),pdu->getPDULength(),0);
\end{lstlisting}

Quando receber a mensagem a forma de uso \'e mostrada abaixo:
\begin{lstlisting}
// receive packet
ret = recvfrom(sockfd,&buffer,BUF_LEN,0,recvAddr,recvAddrLen);
CoapPDU *recvPDU = new CoapPDU((uint8_t*)buffer,ret);
if(recvPDU->validate()) {
    recvPDU->getURI(uriBuffer,URI_BUF_LEN,&recvURILen);
    ...
}
\end{lstlisting}

Por ser uma biblioteca bem simplificada e n\~ao possuir depend\^encias diretas com a implementa\c{c}\~ao da camada de transporte e ser c\'odigo livre, foi escolhida para a implementa\c{c}\~ao teste do trabalho.


\chapter{Proposta}
Este trabalho prop\~oes a implementa\c{c}\~ao de uma biblioteca que utiliza a camada UDP do EPOS para dar suporte ao protocolo CoAP.

O trabalho tamb\'em consiste na implementa\c{c}\~ao de uma aplica\c{c}\~ao para gateway GPRS/Zigbee utilizando o EPOS e um componente de hardware que ser\'a acoplado ao EposMoteII. Este componente esta sendo desenvolvido em paralelo por um colega de laborat\'orio.

O desenvolvimento da aplica\c{c}\~ao no EPOS que ser\'a respons\'avel pelo roteamento de mensagens para Internet utiliza a tecnologia GPRS, provida por um m\'odulo GSM/GPRS da Quectel o M95.

As principais fun\c{c}\~oes deste gateway s\~ao receber os dados da rede de sensores e encaminh\'a-las para um servidor remoto que armazenar\'a essas informa\c{c}\~oes e exibir\'a de forma conveniente para o usu\'ario final.

As fun\c{c}\~oes a serem desenvolvidas na aplica\c{c}\~ao do gateway s\~ao:

\begin{itemize}[noitemsep,topsep=0pt,parsep=0pt,partopsep=0pt]
    \item Configura\c{c}\~ao SMS
    \item Envia mensagem SMS
    \item Recebe mensagem SMS
    \item Configura\c{c}\~ao contexto PDP
    \item Configura\c{c}\~ao GPRS
    \item Configura\c{c}\~ao TCP/IP
    \item Manter uma conex\~ao TCP/IP 
\end{itemize}

\section{Metas}
Entregar um m\'odulo simplificado do procotolo CoAP no primeiro semestre e testar as funcionalidades no modem GSM/GPRS.
Desenvolver a aplica\c{c}\~ao no EPOS no segundo semestre e fazer testes de integra\c{c}\~ao.

\section{Cronograma}
\begin{ganttchart}[
    vgrid, hgrid
    ] {1}{21}
    \gantttitle {Trabalho de Conclus\~ao de Curso}{21} \\
    \gantttitlelist {3,...,12}{2}\\
    \ganttgroup {2013.1} {1}{10}\\
    \ganttbar {Estudar EPOS} {1}{4}\\
    \ganttbar {Estudar EPOSMoteII}{1}{4}\\
    \ganttbar {Estudar modem GPRS}{1}{4}\\
    \ganttbar {Levantar requisitos libcoap no EPOS} {3}{4} \\
    \ganttbar {Levantar requisitos do HW} {1}{6}\\
    \ganttbar {Implementa\c{c}\~ao do CoAP} {3}{10} \\
    \ganttbar {Testar modem GPRS} {4}{10} \\
    \ganttbar {Desenvolvimento do prot\'otipo do HW} {7}{10}\\
    \ganttmilestone {Relat\'orio de TCC I}{10}\\
    \ganttnewline[thick, blue]
    \ganttgroup {2013.2}{11}{20} \\
    \ganttbar {Implementa\c{c}\~ao do Gateway} {11}{18}\\
    \ganttbar {Testes de Integra\c{c}\~ao} {19}{20}\\
    \ganttbar {Documenta\c{c}\~ao do trabalho} {1}{20} \\
    \ganttmilestone {Relat\'orio de TCC II}{20}

    \ganttlink {elem2}{elem7}
    \ganttlink {elem3}{elem7}
    \ganttlink {elem5}{elem7}
    \ganttlink {elem1}{elem11}
    \ganttlink {elem4}{elem11}
    \ganttlink {elem7}{elem10}
    \ganttlink {elem11}{elem12}
    \ganttlink {elem12}{elem11}
    \ganttlink {elem11}{elem13}
    \ganttlink {elem10}{elem13}

\end{ganttchart}

\section{Resultados parciais}

Abaixo est\~ao listadas as atividades conclu\'idas at\'e o momento:
\begin{itemize}
    \item Estudar EPOS.
    \item Estudar EPOSMoteII.
    \item Estudar modem GPRS.
    \item Levantar requisitos libcoap no EPOS.
    \item Levantar requisitos do HW.
    \item Implementa\c{c}\~ao do CoAP: a parte de valida\c{c}\~ao de um pacote CoAP foi feita com TDD e est\'a dispon\'ivel no site: (TODO).
    \begin{itemize}[noitemsep,topsep=0pt,parsep=0pt,partopsep=0pt]
        \item Ao receber uma mensagem de confirma\c{c}\~ao, remove da lista a mensagem que n\~ao havia sido confirmada utilizando o id.
        \item Ao receber uma mensagem confirm\'avel, envia uma mensagem de confirma\c{c}\~ao.
        \item Repassar mensagem para controle de Requisi\c{c}\~ao e Reposta.
        \item Adicionar a lista de mensagem recebidas.
        \item Enviar mensagem n\~ao confirm\'avel.
        \item Enviar mensagem confirm\'avel e adicionar na lista de confirma\c{c}\~oes pendentes.
        \item Reenviar a mensagem que est\'a na lista de confirma\c{c}\~ao pendente e reconfigurar o pr\'oximo reenvio.
    \end{itemize}
    
    \item Testes no m\'ouldo GPRS:
        \begin{itemize}[noitemsep,topsep=0pt,parsep=0pt,partopsep=0pt]
            \item Enviar Mensagem
            \item Receber Mensagem
            \item Criar socket TCP
            \item Enviar mensagem via socket
            \item Receber mensagem via socket
            \item Fazer requisi\c{c}\~ao HTTP para um webserver, foi poss\'ivel utilizando os comandos propriet\'arios do modem.
        \end{itemize}
    \item Desenvolvimento do prot\'otipo do HW.
\end{itemize}

\subsection{Atividades pendentes}
As atividades que ser\~ao feitas ao longo do segundo semestre:
\begin{itemize}
    \item Implementa\c{c}\~ao do Gateway.
    \item Testes de Integra\c{c}\~ao.
    \item Documenta\c{c}\~ao do trabalho.
    \item Relat\'orio de TCC II.
\end{itemize}


\chapter{Considera\c{c}\~oes Finais}
O que \'e este documento?

Objetivo do trabalho


%A aplica\c{c}\~ao desenvolvida no EPOS precisar\'a de um buffer para o recebimento de dados da rede ZigBEE que ser\'a enviado para rede via GPRS. Duas threads, uma produtora que ficar\'a escutando o r\'adio ZigBEE e alimentando num formato de dados ainda n\~ao especificado. Outra consumidora que ser\'a respons\'avel em utlizar estes dados na rede de sensores e encaminh\'a-los pra Internet usando a extens\~ao GPRS do EPOSmote II.


O gateway ZigBEE/GPRS ter\'a a capacidade de fazer a ponte entre a rede de sensores e a Internet, utilizando um modem GPRS. Ser\'a projetado utilizando um mote, um modem GPRS al\'em do circuito necess\'ario para integra\c{c}\~ao dos dois m\'odulos. Ser\'a a parte de hardware do projeto, que ser\'a desenvolvida em paralelo por um colega do laborat\'orio.


\bibliography{bibliografia}{}
\bibliographystyle{ufscThesis/ufsc-alf}

%--------------------------------------------------------
% Elementos pós-textuais

%\anexo
%\chapter{C\'odigo Desenvolvido}
%\lstdefinestyle{customc}{
  belowcaptionskip=1\baselineskip,
  breaklines=true,
  xleftmargin=\parindent,
  language=C++,
  showstringspaces=false,
  basicstyle=\scriptsize\ttfamily,
  keywordstyle=\bfseries\color{green!40!black},
  commentstyle=\itshape\color{blue},
  identifierstyle=\color{blue!20!black},
  stringstyle=\color{red},
}

\subsection{CoAP}

\begin{lstlisting}

#ifndef __coap_packet_h__
#define __coap_packet_h__

#include <coap_pdu.h>

__BEGIN_SYS

class CoapPacket : public CoapPDU
{
    public:
        static const int ackTimeout = 2000000;
        static const double ackRandomFactor = 1.5;
        static const int maxRetransmit = 4;
        static const int nStart = 1;

        CoapPacket(const UDP_Address & to);
        CoapPacket(const UDP_Address & from, const char * data, unsigned len);
        CoapPacket(CoapPDU * pdu);
        CoapPacket();
        ~CoapPacket();

        bool isConfirmable();
        bool isConfirmed();
        bool isFailure() {return false;};
        void setConfirmed();
        void reset();
        virtual void update() {kout << "Packet update" << endl;};
        UDP_Address remote() const;
        int getToken();
    protected:
        void generateNewToken(unsigned int len);
    private:
        static const int _maxPDUSize = 100;
        bool _isConfirmed;
        UDP_Address _remote;
        u8 _pduBuffer[_maxPDUSize];
        Alarm * _alarm;
};

class CoapACK : public CoapPacket
{
    public:
        CoapACK(int id) : CoapPacket() {
            setType(CoapPacket::COAP_ACKNOWLEDGEMENT);
            setCode(CoapPacket::COAP_EMPTY);
            setMessageID(id);
        }
};

class CoapConfirmable : public CoapPacket {
    public:
        CoapConfirmable(CoapPacket::Code code);
        ~CoapConfirmable();
        void update();
        bool isFailure();
        void reset();
    private:
        Alarm * _alarm;
        int _retransmissionCounter;
        int _timeout;
};

__END_SYS

#endif /*__coap_packet_h__*/


#ifndef __coap_request_h__
#define __coap_request_h__

#include <coap_packet.h>

__BEGIN_SYS

class CoapRequest : public CoapConfirmable
{
    public:
        CoapRequest(CoapPacket::Code code, const char* uri);
        virtual void onSuccess();
        virtual void onError();
        static void incomingResponse(CoapPacket * packet);
        bool wasAnswered() { return (_response != 0);}
    private:
        void addResponse(CoapPacket *);
        void addAsPending();
        static CoapRequest * removePendingByToken(int token);
        static unsigned int indexByToken(int token);
        static const int _maxPDUSize = 100;
        static const int _maxPending = 100;
        char _uriBuffer[_maxPDUSize];
        int _uriSize;
        static CoapRequest * _pendingRequests[_maxPending];
        CoapPacket * _response;
};

__END_SYS

#endif /*__coap_request_h__*/


#ifndef __coap_response_h__
#define __coap_response_h__

__BEGIN_SYS

class CoapPacket;

template <class T>
class CoapResponse : public CoapPacket
{
    public:
        CoapResponse(T data, CoapRequest * r);
        ~CoapResponse();
    private:
        T * _data;
};

__END_SYS

#endif /*__coap_response_h__*/


#ifndef __coap_socket_h_
#define __coap_socket_h_

#include <udp.h>
#include <coap_packet.h>
#include <utility/queue.h>

#define COAP_PORT 5683

__BEGIN_SYS

class CoapSocket : public UDP::Socket
{
public:
    CoapSocket(const UDP_Address & local);
    ~CoapSocket();
    void sendPacket(CoapPacket * pdu);
    void sendPacket(const CoapPacket & pdu);
    void received(const UDP_Address & from, const char* msg, unsigned int len);
};

__END_SYS

#endif /* __coap_socket_h_ */


#include <coap_packet.h> 
#include <coap_socket.h>
#include <utility/handler.h>
__BEGIN_SYS

class CoapTransport
{
    public:
    static Function_Handler * dispatcher();
    static void incoming(CoapPacket * in);
    static void outgoing(CoapPacket * out);
    
    private:
    CoapTransport();
    ~CoapTransport();
    void addNonConfirmed(CoapPacket* p);
    static void dispatch();
    CoapSocket * _socket;
    Queue<CoapPacket> * _queue;
    Function_Handler * _handler;
    static CoapTransport * getInstance();
    static CoapTransport * _instance;
};
__END_SYS


#include <utility/random.h>
#include <utility/string.h>
#include <coap_packet.h>
#include <coap_transport.h>

__BEGIN_SYS

/*
 * For a new Confirmable message, the initial timeout is set
 * to a random duration (often not an integral number of seconds)
 * between ACK_TIMEOUT and (ACK_TIMEOUT * ACK_RANDOM_FACTOR) (see
 * Section 4.8)
 * */
CoapPacket::CoapPacket(const UDP_Address & to) : CoapPDU(_pduBuffer, _maxPDUSize, 4),
     _remote(to)
{
    setVersion(1);
    _isConfirmed = false;
    int range = (CoapPacket::ackTimeout * CoapPacket::ackRandomFactor) - CoapPacket::ackTimeout;
    int rand = Pseudo_Random::random() % range;
    setMessageID(rand);
}

CoapPacket::CoapPacket() : CoapPDU(_pduBuffer, _maxPDUSize, 4), _remote(UDP_Address("10.0.2.15:5863"))
{
    setVersion(1);
    _isConfirmed = false;
    int range = (CoapPacket::ackTimeout * CoapPacket::ackRandomFactor) - CoapPacket::ackTimeout;
    int rand = Pseudo_Random::random() % range;
    setMessageID(rand);
}

CoapPacket::CoapPacket(const UDP_Address & from, const char * data, unsigned len) : CoapPDU(_pduBuffer, _maxPDUSize, len),
    _remote(from)
{
    strncpy((char*)_pduBuffer, data, len);
}

CoapPacket::CoapPacket(CoapPDU * pdu) : CoapPDU(pdu->getPDUPointer(), _maxPDUSize, pdu->getPDULength()),
    _remote(UDP_Address("10.0.2.15:5863"))
{
    setMessageID(pdu->getMessageID());
}

void CoapPacket::reset()
{
    CoapPDU::reset();
    _isConfirmed = false;
}

int CoapPacket::getToken()
{
    return atol((char*)getTokenPointer());
}

CoapPacket::~CoapPacket()
{
}

CoapConfirmable::~CoapConfirmable()
{
    if (_alarm) delete _alarm;
    _alarm = 0;
}


void CoapConfirmable::reset()
{
    CoapPacket::reset();
    if (_alarm) delete _alarm;
    _alarm = 0;
}

void CoapPacket::setConfirmed()
{
    _isConfirmed = true;
}

bool CoapPacket::isConfirmed()
{
    return _isConfirmed;
}

UDP_Address CoapPacket::remote() const
{
    return _remote;
}

bool CoapPacket::isConfirmable()
{
    return (getType() == CoapPacket::COAP_CONFIRMABLE);
}

void CoapPacket::generateNewToken(unsigned int len)
{
    if (len > 4) return;
    kout << "size of token: " << len << endl;
    int range = 0xFF << len;
    kout << "range:" << range << endl;
    int rand = Pseudo_Random::random() % range;
    kout << "random:" << rand << endl;
    static const int size = 20;
    char buf[size];
    itoa(rand, buf);
    setToken((u8*)buf, len);
}

void CoapConfirmable::update()
{
    kout << "Confirmable update" << endl;
    if (!isConfirmable()) return;
    _retransmissionCounter++;
    _timeout = _timeout * 2;
    if (_alarm) delete _alarm;
    if (isConfirmed()) {
        kout << "already confirmed!" << endl;
        return;
    }
    kout << "configure new alarm with " << _timeout << endl;
    _alarm = new Alarm(_timeout, CoapTransport::dispatcher(), 1);
}

bool CoapConfirmable::isFailure()
{
    return (_retransmissionCounter > CoapPacket::maxRetransmit);
}
        
CoapConfirmable::CoapConfirmable(CoapPacket::Code code) : CoapPacket() {
    setType(CoapPacket::COAP_CONFIRMABLE);
    setCode(code);
    int range = (CoapPacket::ackTimeout * CoapPacket::ackRandomFactor) - CoapPacket::ackTimeout;
    int rand = Pseudo_Random::random() % range;
    _timeout = CoapPacket::ackTimeout + rand;
    _retransmissionCounter = 0;
    _alarm = 0;
}


__END_SYS


#include <system/config.h>
#include <coap_request.h>
#include <coap_transport.h>

__BEGIN_SYS

CoapRequest * CoapRequest::_pendingRequests[] = {0};

CoapRequest::CoapRequest(CoapPacket::Code code, const char* uri)
    : CoapConfirmable(code)
{
    _uriSize = strlen(uri);
    strncpy(_uriBuffer, uri, _uriSize);
    generateNewToken(4);
    setURI(_uriBuffer, _uriSize);
    CoapTransport::outgoing(this);
    addAsPending();
}

void CoapRequest::incomingResponse(CoapPacket * packet)
{
    CoapRequest * req = removePendingByToken(packet->getToken());
    if (req) req->addResponse(packet);
}

void CoapRequest::addResponse(CoapPacket * packet)
{
    if (packet->isFailure()) {
        onError();
    } else {
        onSuccess();
        _response = packet;
    }
}

void CoapRequest::addAsPending()
{
    unsigned int index = indexByToken(getToken());
    CoapRequest::_pendingRequests[index] = this;
}

unsigned int CoapRequest::indexByToken(int token)
{
    return token % CoapRequest::_maxPending;
}

CoapRequest * CoapRequest::removePendingByToken(int token)
{
    unsigned int index = indexByToken(token);
    CoapRequest * ptr = CoapRequest::_pendingRequests[index];
    CoapRequest::_pendingRequests[index] = 0;
    return ptr;
}

void CoapRequest::onSuccess()
{
    kout << "success!" << endl;
}

void CoapRequest::onError()
{
    kout << "error!" << endl;
}

__END_SYS


#include <alarm.h>
#include <utility/handler.h>
#include <utility/list.h>
#include <udp.h>
#include <coap_request.h>
#include <coap_transport.h>

__BEGIN_SYS

static const int idsSize = 1000;
static int ids[idsSize] = {0};

CoapTransport * CoapTransport::_instance = 0; 

static void updateIds(int id) {
    int pos = id % idsSize;
    ids[pos] = id;
}

static bool checkId(int id) {
    int pos = id % idsSize;
    return (ids[pos] == id);
}

CoapTransport * CoapTransport::getInstance()
{
    if (CoapTransport::_instance == 0) {
        CoapTransport::_instance = new CoapTransport();
    }
    return CoapTransport::_instance;
}

CoapTransport::CoapTransport()
{
    _handler = new Function_Handler(CoapTransport::dispatch);
    _queue = new Queue<CoapPacket>();
    _socket = new CoapSocket(UDP_Address(IP::instance()->address(), COAP_PORT));
}

void CoapTransport::dispatch()
{
    CoapTransport * t = CoapTransport::getInstance();
    if (!t) return;
    Queue<CoapPacket>::Element * e = t->_queue->remove();
    CoapPacket * p = e->object();
    if (!p) return;
    if (p->isFailure()){
        kout << "failure "<< endl;
        if (e) delete e;
        return;
    }
    outgoing(p);
    if (e) delete e;
}

void CoapTransport::addNonConfirmed(CoapPacket * p)
{
    Queue<CoapPacket>::Element * e = new Queue<CoapPacket>::Element(p);
    _queue->insert(e);
}

CoapTransport::~CoapTransport()
{
    delete _handler;
    delete _socket;
    delete[] _queue;
}

void CoapTransport::incoming(CoapPacket * in) {
    if (!in) return;
    CoapTransport * t = CoapTransport::getInstance();
    if (!t) return;
    kout << "INCOMING packet" << endl;
    in->print();
    if (in->getType() == CoapPacket::COAP_ACKNOWLEDGEMENT) {
        kout << "is ACK!" << endl;
        updateIds(in->getMessageID());
    }
    if (in->getType() >= CoapPacket::COAP_CREATED) {
        kout << "is a Response!" << endl;
        CoapRequest::incomingResponse(in);
    }
}

void CoapTransport::outgoing(CoapPacket * out)
{
    if (!out) return;
    kout << "OUTGOING packet" << endl;
    kout << "-------------------------" << endl;
    out->print();
    kout << "-------------------------" << endl;
    CoapTransport * t = CoapTransport::getInstance();
    if (!t) return;
    t->_socket->sendPacket(out);
    if (out->isConfirmable()) {
        kout << "Is confirmable!" << endl;
        if (checkId(out->getMessageID())) {
            kout << "is confirmed!" << endl;
            return;
        }
        t->addNonConfirmed(out);
        out->update();
    }
}

Function_Handler * CoapTransport::dispatcher()
{
    CoapTransport * t = CoapTransport::getInstance();
    if (!t) return 0;
    return _instance->_handler;
}

__END_SYS


#include <coap_transport.h>
#include <coap_socket.h>

__BEGIN_SYS

CoapSocket::CoapSocket(const UDP_Address & local) : UDP::Socket(local, UDP::Address(Traits<IP>::BROADCAST, COAP_PORT))
{
}

CoapSocket::~CoapSocket()
{
}

void CoapSocket::received(const UDP_Address& from, const char* data, unsigned int len)
{
	CoapPacket *packet = new CoapPacket(from, data, len);

	if(!packet->validate()) return;
	
	packet->print();

	if (packet->getType() == CoapPacket::COAP_ACKNOWLEDGEMENT) {
	    kout << "is ack!" << endl;
    }

    CoapTransport::incoming(packet);
}

void CoapSocket::sendPacket(CoapPacket * pdu)
{
    if (!pdu) return;
    remote(pdu->remote());
    send(reinterpret_cast<char*>(pdu->getPDUPointer()), pdu->getPDULength());
}

__END_SYS
\end{lstlisting}


\subsection{Aplica\c{c}\~ao WEB}


\end{document}
